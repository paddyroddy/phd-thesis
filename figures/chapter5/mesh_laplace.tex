\begin{figure}[htpb]
    \centering\capstart{}
    \begin{overpic}
        [width=0.47\textwidth]{mesh_laplacian.pdf}
        \small
        \put(101,53){\(h\)}
        \put(48,32){\(i\)}
        \put(69,83.5){\(j\)}
        \put(28,72.5){\(k\)}
        \put(38,69){\(\alpha_{i j}\)}
        \put(85,56){\(\beta_{i j}\)}
        \put(35.5,44.5){\color{red}\(a_{i}\)}
        \put(50,67){\color{red}\(a_{i j k}\)}
        \put(65,60){\(\ell_{i j}\)}
    \end{overpic}
    \caption[
        The triangular mesh discretisation of a two-dimensional manifold
    ]{
        The triangular mesh discretisation of a two-dimensional manifold.
        A triangular face is highlighted in red which connects the \(i\), \(j\) and \(k\) vertices, with the corresponding area represented as \(a_{i j k}\).
        Similarly, the vertex weight \(a_{i}\) is the red hexagonal shape which is effectively one-third of the area of the surrounding faces, \ie{} \cref{eq:chapter5_vertex_weight}.
        The \(i\) and \(j\) vertices are connected by an edge of length \(\ell_{i j}\), and the corresponding interior angles at the vertices \(k\) and \(h\) are given by \(\alpha_{i j}\) and \(\beta_{i j}\) respectively.
        These interior angles (in grey) appear in the expression for the cotangent Laplacian, \cf{} \cref{eq:chapter5_cotangent_laplace}.
    }\label{fig:chapter5_mesh_laplace}
\end{figure}
