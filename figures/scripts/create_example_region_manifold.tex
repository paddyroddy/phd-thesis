% Inspired by Fionn Fitzmaurice's differential geometry notes
% https://gist.github.com/fionn/f27ddeadba97f3dabbf97527b05b33e5
\documentclass[tikz]{standalone}

\usetikzlibrary{decorations,hobby,patterns}

\pgfkeys{/pgf/decoration/.cd,
	stipple density/.store in=\pgfstippledensity,
	stipple density=.1,
	stipple scaling function/.store in=\pgfstipplescalingfunction,
	stipple scaling function=sin(\pgfstipplex*180)*0.875+0.125, % chktex 36
	stipple radius/.store in=\pgfstippleradius,
	stipple radius=0.25pt
}
\pgfdeclaredecoration{stipple}{draw}{
	\state{draw}[width=\pgfdecorationsegmentlength]{%
		\pgfmathparse{\pgfdecoratedcompleteddistance/\pgfdecoratedpathlength}%
		\let\pgfstipplex=\pgfmathresult%
		\pgfmathparse{int(\pgfstippledensity*100)}% chktex 36
		\let\pgfstipplen=\pgfmathresult%
		\pgfmathloop%
		\ifnum\pgfmathcounter<\pgfmathresult\relax%
		\pgfpathcircle{%
			\pgfpoint{(rnd)*\pgfdecorationsegmentlength}%
			{(\pgfstipplescalingfunction)*(rnd^4)*\pgfdecorationsegmentamplitude+\pgfstippleradius}}% chktex 36
		{\pgfstippleradius}%
		\repeatpgfmathloop%
	}
}

\tikzset{
	stipple/.style={
		decoration={stipple, segment length=2pt, #1},
		decorate,
		fill
	}
}

\begin{document}
	\begin{tikzpicture}
		\begin{scope}
			\draw [postaction={stipple={amplitude=0.5cm}},
				postaction={stipple={amplitude=0.25cm}}]
			(0, 0) .. controls ++(65:-1.5) and ++(95:1) .. (-1.5, -4)
			(-1.5, -4) .. controls ++(195:-1.3) and ++(195:1) .. (3, -4)
			(3, -4) .. controls ++(95:2) and ++(95:-1) .. (4, 0)
			node[below right] {$\mathcal{M}$}
			(4, 0) .. controls ++(195:1) and ++(195:-1) .. (0, 0);
		\end{scope}
		\draw [use Hobby shortcut, % .. syntax
			closed=true, % connects region outline
			pattern=north east lines] % direction of lines
		(0.7,-2.3) .. (1.2,-2.6) .. (2.3,-2.2) .. (2,-1.8) .. (1.2,-1.3) .. (0.7,-1.4) .. (0.8,-1.9);
		\node at (1.8,-1.3) {\(R\)};
	\end{tikzpicture}
\end{document}
