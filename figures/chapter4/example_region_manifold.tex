% Inspired by Fionn Fitzmaurice's differential geometry notes
% https://gist.github.com/fionn/f27ddeadba97f3dabbf97527b05b33e5
\begin{figure}[htp]
	\centering\capstart{}
	\resizebox{0.75\textwidth}{!}{%
		\begin{tikzpicture}
			\begin{scope}
				\draw [postaction={stipple={amplitude=0.5cm}},
					postaction={stipple={amplitude=0.25cm}}]
				(0, 0) .. controls ++(65:-1.5) and ++(95:1) .. (-1.5, -4)
				(-1.5, -4) .. controls ++(195:-1.3) and ++(195:1) .. (3, -4)
				(3, -4) .. controls ++(95:2) and ++(95:-1) .. (4, 0)
				node[below right] {$\mathcal{M}$}
				(4, 0) .. controls ++(195:1) and ++(195:-1) .. (0, 0);
			\end{scope}
			\draw [use Hobby shortcut, % .. syntax
				closed=true, % connects region outline
				pattern=north east lines] % direction of lines
			(0.7,-2.3) .. (1.2,-2.6) .. (2.3,-2.2) .. (2,-1.8) .. (1.2,-1.3) .. (0.7,-1.4) .. (0.8,-1.9);
			\node at (1.8,-1.3) {\(R\)};
		\end{tikzpicture}%
	}
	\caption[
		An example region of a manifold
	]{
		In various domains, data are observed on a partial region of a manifold \(\mathcal{M}\), such as \(R\).
	}\label{fig:chapter4_region}
\end{figure}
