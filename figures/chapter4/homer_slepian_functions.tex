\begin{figure}[htp]
	\centering
	\subfloat[\(\mesh{S_{1}},\ \mu=1.00\)]
	{\includegraphics[trim={101 0 3 3},clip,width=.33\textwidth]{slepian_homer_rank0_lam1-000000e00_zoom.pdf}} % chktex 8
	\hfill
	\subfloat[\(\mesh{S_{10}},\ \mu=1.00\)]
	{\includegraphics[trim={101 0 3 3},clip,width=.33\textwidth]{slepian_homer_rank9_lam1-000000e00_zoom.pdf}} % chktex 8
	\hfill
	\subfloat[\(\mesh{S_{25}},\ \mu=1.00\)]
	{\includegraphics[trim={101 0 3 3},clip,width=.33\textwidth]{slepian_homer_rank24_lam1-000000e00_zoom.pdf}} % chktex 8
	\newline
	\subfloat[\(\mesh{S_{50}},\ \mu=1.00\)]
	{\includegraphics[trim={101 0 3 3},clip,width=.33\textwidth]{slepian_homer_rank49_lam1-000000e00_zoom.pdf}} % chktex 8
	\hfill
	\subfloat[\(\mesh{S_{100}},\ \mu=1.00\)]
	{\includegraphics[trim={101 0 3 3},clip,width=.33\textwidth]{slepian_homer_rank99_lam1-000000e00_zoom.pdf}} % chktex 8
	\hfill
	\subfloat[\(\mesh{S_{200}},\ \mu=1.00\)]
	{\includegraphics[trim={101 0 3 3},clip,width=.33\textwidth]{slepian_homer_rank199_lam1-000000e00_zoom.pdf}} % chktex 8
	\caption{
		The Slepian functions of the Homer head region \(\mesh{\slepian{S}}\) for \(p \in \set{1, 10, 25, 50, 100, 200}\) shown left-to-right, top-to-bottom.
		The corresponding eigenvalue \(\slepian{\mu}\) is a measure of the concentration within the given region \(R\) which remain \(\almost{1}\) for many \(p\) values before decreasing towards zero.
	}\label{fig:chapter4_slepian_functions}
\end{figure}
