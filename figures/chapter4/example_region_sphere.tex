\begin{figure}[htpb]
\centering\capstart{}
\resizebox{.5\textwidth}{!}{%
\begin{tikzpicture}
\draw [use Hobby shortcut, % .. syntax
closed=true, % connects region outline
pattern=north east lines, % direction of pattern
even odd rule] % required for region to have different pattern
(0,0) circle (2cm)
(-0.2,0.3) .. (0.2,0.2) .. (1.1,0.6) .. (1.1,1) .. (0.3,1.4) .. (-0.4,1.2) .. (-0.3,0.8); % region outline
\draw (0,0) circle (2cm) % perimeter
(-2,0) arc (180:360:2 and 0.6); % front equator
\draw[dashed] (2,0) arc (0:180:2 and 0.6); % rear equator
\fill [fill=black] (0,0) circle (1pt); % central dot
\node at (0.4,0.8) {\(R\)}; % label
\end{tikzpicture}%
}
\caption[
An example region on the sphere
]{
In many application domains, data are observed on a partial region of the sphere only, such as \(R\).
}\label{fig:chapter4_region}
\end{figure}
