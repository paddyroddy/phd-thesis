\chapter{Background}\label{sec:chapter2}

In this chapter the mathematical background that subsequent chapters rely on is introduced.

\section{Mathematical Preliminaries}

\subsection{Legendre Polynomials}

\subsection{Associated Legendre Polynomials}

\subsection{Spherical Harmonics}

The spherical harmonics \(\pixel{\harmonic{Y}}\), with integers \(\ell \geq 0\) and \(m \leq \abs{\ell}\), form the complete set of orthogonal basis functions for the Hilbert space \(\hilbert{\twoSphere}\).
Here \(\omega=(\theta,\phi)\) parameterise a point on the unit sphere, where \(\theta \in \interval{0}{\pi}\) is the colatitude, \(\phi \in \interval[open right]{0}{2\pi}\) is the longitude, and \(\dd{\Omega(\omega)} = \sin{\theta} \dd{\theta} \dd{\phi}\) is the usual rotation invariant measure on the 2-sphere \(\twoSphere\).
The spherical harmonics arise as the solutions to the Laplacian in spherical coordinates; in which a separation of variables leads to solutions in terms of the associated Legendre polynomials and the complex exponentials.
The spherical harmonics are then defined by combining these respective parts:
%
\begin{equation}
	\pixel{\harmonic{Y}}
	%
	= {(-1)}^{m} \sqrt{\factor \frac{(\ell-m)!}{(\ell+m)!}} P^{m}_{\ell}(\cos{\theta}) \exp(i m\phi),
\end{equation}
%
where the \emph{Condon-Shorley} phase convention is followed with the addition of the \({(-1)}^{m}\) phase factor to ensure conjugate symmetry, \ie{}
%
\begin{equation}
    \pixel{\conj{\harmonic{Y}}}
    %
    = {(-1)}^{m} \pixel{Y_{\ell(-m)}}.
\end{equation}
%
A phase factor is incorporated in the definition of the spherical harmonics directly rather than in the associated Legendre polynomials as some others do, following the convention of~\cite{Brink1993}.
The spherical harmonics satisfy the following orthonormality and completeness properties, respectively:
%
\begin{equation}
    \integrateSphere{\omega} \pixel{\harmonic{Y}} \pixel[']{\conj{\harmonic[']{Y}}}
    %
    = \delta_{\ell\ell'} \delta_{mm'}
\end{equation}
%
and
%
\begin{align}
    \sum\limits_{\ell=0}^{\infty} \sum\limits_{m=-\ell}^{\ell} \pixel{\harmonic{Y}} \pixel[']{\conj{\harmonic[']{Y}}}
    %
    &= \delta(\omega - \omega') \nonumber{} \\
    %
    &= \delta(\phi - \phi') \delta(\cos{\theta} - \cos{\theta'}),
\end{align}
%
where \(\delta(\omega)\) is the Dirac delta function.
Note a further useful relation for the spherical harmonics, the \emph{addition theorem}:
%
\begin{equation}
	\sum\limits_{m=-\ell}^{\ell} \pixel{\harmonic{Y}} \pixel[']{\conj{\harmonic{Y}}}
	%
	= \factor P_{\ell}(\omega \cdot \omega').
\end{equation}


\subsection{Wigner Functions}

\subsection{Rotations on the Sphere}

\subsection{Harmonic Expansions}
