\chapter{Background}\label{sec:chapter2}

In this chapter the mathematical background that later chapters rely on is introduced.

\section{Cosmology}

\subsection{Geometry and Dynamics}

\subsubsection{Metric}

In considering the universe on large-scales one makes the following assumptions
%
\begin{enumerate}[(i),nosep,left=\parindent]
	\item the universe is isotropic,
	\item the universe if homogeneous, and
	\item general relativity is the correct description of gravity at all scales.
\end{enumerate}
%
The cosmic microwave background (CMB) strongly supports these first two assumptions as its temperature is uniform to one part in \(\num{e5}\) and the fluctuations in temperature are statistically isotropic.
The last assumption holds well on solar system scales and on larger scales when just \emph{dark matter} and \emph{dark energy} are present in the standard model.
These three assumptions mean that the metric takes the form
%
\begin{equation}\label{eq:original_metric}
	\dd{s^{2}}
    %
    = g_{\mu\nu} \dd{X^{\mu}} \dd{X^{\nu}}
    %
    = -\dd{t^{2}} + a^{2}(t) \dd{\ell^{2}}
\end{equation}
%
where \(a(t)\) is the scale factor and
%
\begin{equation}
	\dd{\ell^{2}}
    %
    = \gamma_{i j} \dd{x^{i}} \dd{x^{j}}
\end{equation}
%
is the constant curvature 3-metric.
This metric can take three forms: flat space, positive curvature and negative curvature, which are described below.

\paragraph{Flat Space}

The line element of three-dimensional Euclidean space \(\mathbb{E}^{3}\) is just
%
\begin{equation}
    \dd{\ell^{2}}
    %
    = \dd{\vb{x}^{2}}
    %
    = \delta_{i j} \dd{x^{i}} \dd{x^{j}}
\end{equation}
%
which is invariant under spatial translations and rotations.

\paragraph{Positive Curvature}

One can represent a 3-space with constant positive curvature as a 3-sphere \(\mathbb{S}^{3}\) embedded in four-dimensional Euclidean space \(\mathbb{E}^{4}\),
%
\begin{equation}
    \dd{\ell^{2}}
    %
    = \dd{\vb{x}^{2}} + \dd{u^{2}},\ \abs{\vb{x}}^{2} + u^{2}
    %
    = R^{2}
\end{equation}
%
where \(R\) is the radius of the 3-sphere.
Homogeneity and isotropy of the surface of the 3-sphere result from the symmetry of the line element under four-dimensional rotations.

\paragraph{Negative Curvature}

A hyperboloid \(\mathbb{H}^{3}\) embedded in four-dimensional Lorentzian space \(\mathbb{R}^{1,3}\) can represent a 3-space with constant negative curvature as
%
\begin{equation}
    \dd{\ell^{2}}
    %
    = \dd{\vb{x}^{2}} - \dd{u^{2}},\ \abs{\vb{x}}^{2} - u^{2}
    %
    = -R^{2}
\end{equation}
%
where \(R^{2}\) is an arbitrary constant.
Homogeneity and isotropy of the surface of the 3-sphere result from the symmetry of the line element under four-dimensional pseudo-rotations (\ie{} Lorentz transformations, with \(u\) acting as time).

Rescaling the coordinates, \(\vb{x} \rightarrow R\vb{x}\) and \(u \rightarrow Ru\), eliminates the dummy variable \(u\) such that the 3-metric becomes
%
\begin{equation}
	\dd{\ell^{2}}
    %
    = R^{2} (\dd{\vb{x}^{2}} \pm \dd{u^{2}}),\
    %
    \vb{x}^{2} \pm u^{2}
    %
    = \pm 1.
\end{equation}
%
Now from the constraint note \(u\dd{u} \mp \vb{x}\dd{\vb{x}}\) which, substituting in \cref{eq:original_metric}, results in
%
\begin{equation}\label{eq:frw_metric}
	\dd{s^{2}}
    %
    = -\dd{t^{2}} + a^{2}(t) \bigg( \frac{\dd{r^{2}}}{1-Kr^{2}} + r^{2}\dd{\Omega^{2}} \bigg),\
    %
    K =
	%
	\begin{cases}
		\mathbin{\phantom{-}}1 & \mathbb{S}^{3} \\
        %
		\mathbin{\phantom{-}}0 & \mathbb{E}^{3} \\
        %
		-1                     & \mathbb{H}^{3}
	\end{cases}
\end{equation}
%
where the scale factor absorbs the radius of curvature of the 3-metric \(a(t)R \rightarrow a(t)\).
In defining the Friedmann-Robertson-Walker (FRW) metric \cref{eq:frw_metric}, the comoving curvature \(K\) takes care of the three topologies.

\subsubsection{Kinematics}

Assuming gravity is the sole force to act on a particle then it moves along a \emph{geodesic}.
The \emph{four-velocity} of a massive particle is
%
\begin{equation}
	U_{\mu}
    %
    = \dv{X^{\mu}}{s}.
\end{equation}
%
A geodesic is a curve which extremises the proper time \(\Delta s/c\) between two points in spacetime. The \emph{geodesic equation} defines the motion of the particle through
%
\begin{equation}\label{eq:geodesic_equation}
	\dv{U^{\mu}}{s} + \christoffel{\mu}{\alpha}{\beta} U^{\alpha} U^{\beta}
    %
    = 0
\end{equation}
%
where \(\christoffel{\mu}{\alpha}{\beta}\) are the \emph{Christoffel symbols}
%
\begin{equation}
	\christoffel{\mu}{\alpha}{\beta}
    %
    = \frac{1}{2} g^{\mu\lambda}
    %
    (\partial_{\alpha}g_{\beta\lambda}
    %
    + \partial_{\beta}g_{\alpha\lambda}
    %
    - \partial_{\lambda}g_{\alpha\beta})
\end{equation}
%
through the use of the notation \(\partial_{\mu} = \partial/\partial X^{\mu}\).
\(U_{\mu}\) advances from a tangent vector to the geodesic to a vector field in the neighbourhood of the particle
%
\begin{equation}
	\frac{U^{\mu}}{\dd{s}}
    %
    = \dv{X^{\alpha}}{s} \frac{U^{\mu}}{\dd{X^{\alpha}}}
    %
    = U^{\alpha} \partial_{\alpha} U^{\mu}
\end{equation}
%
such that the geodesic equation \cref{eq:geodesic_equation} becomes
%
\begin{equation}\label{eq:geodesic_covariant}
	U^{\alpha} (\partial_{\alpha} U^{\mu} + \christoffel{\mu}{\alpha}{\beta} U^{\beta})
    %
    = U^{\alpha} \nabla_{\alpha} U^{\mu}
    %
    = 0
\end{equation}
%
where \(\nabla_{\alpha}\) is the covariant derivative.
Using the \emph{four-momentum} of the particle \cref{eq:geodesic_covariant} simplifies to
%
\begin{equation}
	P^{\alpha} \partial_{\alpha} P^{\mu}
    %
    = -\christoffel{\mu}{\alpha}{\beta} P^{\alpha} P^{\beta}
\end{equation}
%
which also works for massless particles.

\subsubsection{Redshift}

The wavelength of light is inversely proportional to the photon momentum \(\lambda = h/p\) and the momentum of a photon evolves as \(p \propto {a(t)}^{-1}\), hence the wavelength evolves as \(\lambda \propto a(t)\).
Light emitted at time \(t_{1}\) with wavelength \(\lambda_{1}\) observed at \(t_{0}\) has wavelength
%
\begin{equation}
	\lambda_{0}
    %
    = \frac{a(t_{0})}{a(t_{1})} \lambda_{1}.
\end{equation}
%
Since \(a(t_{0}) > a(t_{1})\), the wavelength of light increases \(\lambda_{0} > \lambda_{1}\).
The fractional change of light defines the redshift
%
\begin{equation}
	z
    %
    = \frac{\lambda_{0} - \lambda_{1}}{\lambda_{1}}
\end{equation}
%
and if one rescales the scale factor such that \(a_{0}=1\) then
%
\begin{equation}
	1 + z
    %
    = \frac{1}{a}.
\end{equation}
%
For nearby sources \(a(t)\)
%
\begin{equation}
	a(t)
    %
    = (1 - (t_{0} - t)H_{0} + \ldots)
\end{equation}
%
where \(H_{0}\) is the Hubble parameter today.
By setting the distance \(d = t_{0} - t\) to first order, one recovers Hubble's law
%
\begin{equation}
	z
    %
    = H_{0}d.
\end{equation}

\subsubsection{Dynamics}

The Einstein equation establishes the evolution of the scale factor
%
\begin{equation}\label{eq:einstein_tensor}
	G_{\mu\nu}
    %
    = 8\pi G T_{\mu\nu},
\end{equation}
%
which relates the Einstein tensor \(G_{\mu\nu}\) to the energy-momentum tensor \(T_{\mu\nu}\).
Due to the assumptions of homogeneity and isotropy the energy-momentum tensor must take the form of a perfect fluid
%
\begin{equation}\label{eq:energy_momentum}
	T_{\mu\nu}
    %
    = (\rho + P) U_{\mu} U_{\nu} + P g_{\mu\nu}.
\end{equation}
%
The conservation of the energy-momentum tensor
%
\begin{equation}
	\nabla_{\mu} T^{\mu}_{\nu}
    %
    = 0
\end{equation}
%
implies the continuity equation
%
\begin{equation}\label{eq:continuity}
	\dot{\rho} + 3\frac{\dot{a}}{a}(\rho + P)
    %
    = 0.
\end{equation}
%
Assuming a constant equation of state
%
\begin{equation}
	w
    %
    = \frac{\rho}{P}
\end{equation}
%
\cref{eq:continuity} becomes
%
\begin{equation}\label{eq:density_equation}
	\rho \propto a^{-3(1+w)}.
\end{equation}
%
The universe comprises a mixture of different matter components.
This includes cold dark matter (\(w=0\)), radiation (\(w=1/3\)) and vacuum energy (\(w=-1\)).
The solutions to \cref{eq:density_equation} are
%
\begin{equation}
	\rho =
	%
	\begin{cases}
		a^{-3} & \text{matter, \ie{} cold dark matter and baryons};        \\
        %
		a^{-4} & \text{radiation, \ie{} photons, neutrinos and gravitons}; \\
        %
		a^{0}  & \text{and dark energy, \ie{} dark energy or something else}.
	\end{cases}
\end{equation}

The non-zero components of the Einstein tensor \cref{eq:einstein_tensor} are \(G_{00}\) and \(G_{i j}\) respectively, which implies
%
\begin{equation}
    3\Bigg[ {\bigg(\frac{\dot{a}}{a}\bigg)}^{2} + \frac{k}{a^{2}} \Bigg]
    %
    = 8\pi G\rho,
\end{equation}
%
and
%
\begin{equation}
    -\Bigg[ 2\frac{\ddot{a}}{a} + {\bigg(\frac{\dot{a}}{a}\bigg)}^{2} + \frac{k}{a^{2}} \Bigg]
    %
    = 8\pi G P.
\end{equation}
%
Combining these with the energy-momentum tensor \cref{eq:energy_momentum}, retrieves the \emph{Friedmann equations}.
These are the fundamental equations governing the evolution of the universe:
%
% resume numbering after Friedmann equations
\addtocounter{equation}{-1}
%
\begin{subequations}
	\begin{align}
		{\bigg(\frac{\dot{a}}{a}\bigg)}^{2}  & = \frac{8\pi G}{3} \rho - \frac{K}{a^{2}} \tag{F1} \label{eq:F1} \\
		%
		\frac{\ddot{a}}{a}                    & = -\frac{4\pi G}{3}(\rho + 3P) \tag{F2}                          \\
		%
		(\dot{F1}) + (F2) \implies \dot{\rho} & = -3\frac{\dot{a}}{a}(\rho + P). \tag{F3}
	\end{align}
\end{subequations}
%
The first Friedmann equation is often defined through the Hubble parameter
%
\begin{equation}
	H^{2}
    %
    = \frac{8\pi G}{3} \rho - \frac{K}{a^{2}},
\end{equation}
%
where \(K=0\) results in the critical density for a flat universe
%
\begin{equation}
	\rho_{\text{critical},0}
    %
    = \frac{3H_{0}^{2}}{8\pi G},
\end{equation}
%
in which the subscript \(0\) denotes quantities evaluated today.
Define the fractional density as
%
\begin{equation}
	\Omega_{X}
    %
    = \frac{\rho_{X}}{\rho_{\text{critical},0}},
\end{equation}
%
such that
%
\begin{equation}
    \sum\limits_{i}\Omega_{i}
    %
    = 1
\end{equation}
%
defines a flat universe.
The Friedmann equations now become
%
\begin{equation}
	H^{2}
    %
    = H_{0}^{2} (\Omega_{r}a^{-4} + \Omega_{m}a^{-3} + \Omega_{k}a^{-2} + \Omega_{\Lambda}),
\end{equation}
%
where one drops the subscript \(0\), and uses the normalisation of the scale factor \(a_{0}=1\).
The curvature density parameter is
%
\begin{equation}
    \Omega_{k}
    %
    = -\frac{a}{{(aH_{0})}^{2}}.
\end{equation}
%
Such that \cref{eq:F1} becomes
%
\begin{equation}
	{\bigg(\frac{\dot{a}}{a}\bigg)}^{2}
    %
    = H_{0}^{2} \Omega a^{-3(1+w)}
\end{equation}
%
and solve for the scale factor in each era:
%
\begin{equation}
	a(t) \propto
	%
	\begin{cases}
		t^{1/2}    & \text{radiation dominated}    \\
		%
		t^{1/3}    & \text{matter dominated}       \\
		%
		\exp(H_{0}t) & \text{dark energy dominated}.
	\end{cases}
\end{equation}
