\chapter{Sifting Convolution on the Sphere}\label{sec:chapter3}

A novel spherical convolution is defined through the sifting property of the Dirac delta on the sphere.
The so-called sifting convolution is defined by the inner product of one function with a translated version of another, that follows by the adoption of an alternative translation operator on the sphere.
This translation operator follows by analogy with the Euclidean translation when viewed in harmonic space.
The sifting convolution satisfies a variety of desirable properties that are lacking in alternate definitions, namely: it supports directional kernels; it has an output that remains on the sphere; and, is efficient to compute.
An illustration of the sifting convolution on a topographic map of the Earth demonstrates that it supports directional kernels to perform anisotropic filtering, while its output remains on the sphere.
This chapter is adapted from~\cite{Roddy2021}, in which I was the lead author.

\section{Introduction}

Many fields in science and engineering measure data on spherical manifolds, such as computer graphics~\cite{Ramamoorthi2004}, planetary science~\cite{Turcotte1981}, geophysics~\cite{Simons2006}, quantum chemistry~\cite{Choi1999}, cosmology~\cite{Bennett1996}, and computer vision~\cite{Cohen2018,Esteves2020,Cobb2021}.
Possible extensions to signal processing techniques developed in the Euclidean domain may be transferred to the spherical domain.
The convolution is an important signal processing technique between two signals defined on the 2-sphere, which is central to filtering --- an integral part of spherical analyses.

Many definitions of spherical convolutions exist in the literature.
A spherical convolution operator would ideally exhibit a variety of desirable properties --- such a spherical convolution would accept directional inputs (\ie{} functions that are \emph{not} invariant under azimuthal rotation), whilst having the output remain on the sphere.
Moreover, the convolution would be efficient to compute.
Existing convolutions such as the isotropic convolution (\eg{}~\cite{McEwen2007,Wei2011,Kennedy2011}) and the left convolution~\cite{Kennedy2011,Driscoll1994} restrict themselves to an axisymmetric kernel (\ie{} kernels that are invariant under azimuthal rotation).
The directional convolution has an output which is not on the sphere (\eg{}~\cite{McEwen2007,Wandelt2001}).
Lastly, the commutative anisotropic convolution~\cite{Sadeghi2012,Khalid2012} and the directional convolution are computationally demanding.
No existing spherical convolution satisfies all three desirable properties.

This chapter presents an alternative spherical convolution, the sifting convolution, defined through the sifting property of the Dirac delta --- in analogy to the Euclidean definition.
The convolution is anisotropic in nature and supports directional kernels.
The output remains on the sphere, even when both inputs are directional.
Moreover, the convolution is efficient to compute and is commutative up to a complex conjugate.

The remainder of this chapter is as follows.
\cref{sec:chapter3_preliminaries} includes some mathematical preliminaries and reviews existing spherical convolutions in the literature.
\cref{sec:chapter3_sifting_convolution} introduces the proposed sifting convolution.
\cref{sec:chapter3_numerical_illustration} presents a demonstration of the convolution with a directional kernel.
Lastly, \cref{sec:chapter3_conclusions} sets out some concluding remarks.

\section{Mathematical Background and Problem Formulation}\label{sec:chapter3_preliminaries}

\subsection{Mathematical Preliminaries}

A brief review of the mathematical preliminaries required for this chapter is presented here.
Recall also \cref{sec:chapter2_notation,sec:chapter2_spherical_harmonics,sec:chapter2_rotations_sphere}.

\subsubsection{Dirac Delta}

In harmonic space the Dirac delta on the sphere rotated to some \(\omega'=(\theta',\phi')\) is
%
\begin{equation}
    \pixel{\delta_{\omega'}}
    %
    = \harmonicSum \pixel[']{\conj{\harmonic{Y}}} \pixel{\harmonic{Y}},
\end{equation}
%
which can be shown to be real
%
\begin{align}\label{eq:chapter3_dirac_delta_real}
    \pixel{\delta_{\omega'}}
    %
     & = \harmonicSum \pixel[']{\conj{\harmonic{Y}}} \pixel{\harmonic{Y}} \nonumber{}                           \\
    %
     & = \harmonicSum \phaseFactor \pixel[']{Y_{\ell(-m)}} \phaseFactor \pixel{\conj{Y_{\ell(-m)}}} \nonumber{} \\
    %
     & = \sum\limits_{\ell m'} \pixel[']{Y_{\ell m'}} \pixel{\conj{Y_{\ell m'}}} \nonumber{}                    \\
    %
     & = \pixel{\conj{\delta_{\omega'}}},
\end{align}
%
where the second line follows by \cref{eq:chapter2_spherical_harmonic_phase}.
The Dirac delta on the sphere is normalised
%
\begin{equation}
    \integrateSphere{\omega} \pixel{\delta_{\omega'}}
    %
    = 1,
\end{equation}
%
as
%
\begin{align}
    \integrateSphere{\omega} \pixel{\delta_{\omega'}}
    %
     & = \integrateSphere{\omega} \harmonicSum \pixel[']{\conj{\harmonic{Y}}} \pixel{\harmonic{Y}} \nonumber{}                            \\
    %
     & = \harmonicSum \pixel[']{\conj{\harmonic{Y}}} \integrateSphere{\omega} \pixel{\harmonic{Y}} \pixel{Y_{00}} \sqrt{4\pi} \nonumber{} \\
    %
     & = \sqrt{4\pi} \harmonicSum \pixel[']{\conj{\harmonic{Y}}} \delta_{\ell0} \delta_{m0} \nonumber{}                                   \\
    %
     & = \sqrt{4\pi} \pixel[']{\conj{Y_{00}}} \nonumber{}                                                                                 \\
    %
     & = 1,
\end{align}
%
where the second and third lines follow by substituting \(\sqrt{4\pi}\pixel{Y_{00}}=1\) (\cf{} \cref{eq:chapter2_spherical_harmonic}) and the orthonormality of the spherical harmonics respectively.
The sifting property of the Dirac delta is
%
\begin{equation}
    \integrateSphere{\omega} \pixel{\delta_{\omega'}} \pixel{f}
    %
    = \pixel[']{f},
\end{equation}
%
as
%
\begin{align}
    \integrateSphere{\omega} \pixel{\delta_{\omega'}} \pixel{f}
    %
     & = \integrateSphere{\omega} \pixel{\conj{\delta_{\omega'}}} \pixel{f} \nonumber{}                                                                                \\
    %
     & = \integrateSphere{\omega} \harmonicSum \pixel[']{\harmonic{Y}} \pixel{\conj{\harmonic{Y}}} \harmonicSum['] \harmonic[']{f} \pixel{\harmonic[']{Y}} \nonumber{} \\
    %
     & = \harmonicSum \harmonic{f} \pixel[']{\harmonic{Y}} \nonumber{}                                                                                                 \\
    %
     & = \pixel[']{f},
\end{align}
%
where the first line holds due to the reality of the Dirac delta \cref{eq:chapter3_dirac_delta_real}.

\subsection{Spherical Convolutions}\label{sec:chapter3_spherical_convolutions}

The conventional convolution between two functions on two-dimensional Euclidean space \(\mathbb{R}^{n}\) is
%
\begin{equation}
    (f \star g)(x)
    %
    = \int\limits_{\mathbb{R}^{2}} \dd{y} f(x-y) g(y),
\end{equation}
%
where \(x,\ y \in \mathbb{R}^{n}\).
The convolution is commutative
%
\begin{equation}
    f \star g
    %
    = g \star f.
\end{equation}
%
A spherical counterpart of the convolution is required for functions defined on the sphere.
Alternative definitions of such a convolution exist in the literature but, while already useful, lack certain desirable properties.
The properties desired in the spherical extension of the convolution include:
%
\begin{enumerate}[(i),nosep,left=\parindent]
    \item the support of directional kernels;
    \item an output that remains on the sphere; and
    \item efficient computation.
\end{enumerate}
%
A convolution is considered computationally efficient here if its computational cost is no greater than the cost of fast spherical harmonic transforms, \ie{} \(\mathcal{O}(L^{3})\) (\eg{}~\cite{Driscoll1994,McEwen2011}).
Formulations of spherical convolutions exist in the literature, but none satisfy all properties.
A summary of existing spherical convolutions and their properties follows.

\subsubsection{Isotropic Convolution}

In real space, the isotropic convolution (\eg{}~\cite{McEwen2007,Wei2011,Kennedy2011}) is
%
\begin{equation}
    \pixel{(f \odot g)}
    %
    = \integrateSphere{\omega'} \pixel[']{f} \pixel[']{\conj{(\rotation{\omega}g)}},
\end{equation}
%
which in harmonic space becomes (\eg{}~\cite{McEwen2007})
%
\begin{equation}\label{eq:chapter3_isotropic_harmonic}
    \harmonic{(f \odot g)}
    %
    = \sqrt{\inverseFactor} \harmonic{f} \conj{\axiHarmonic{g}},
\end{equation}
%
as
%
\begin{align}
    \pixel{(f \odot g)}
    %
     & = \integrateSphere{\omega'} \pixel[']{f} \pixel[']{\conj{(\rotation{\omega}g)}} \nonumber{}                                                                                           \\
    %
     & = \integrateSphere{\omega'} \harmonicSum \harmonic{f} \pixel[']{\harmonic{Y}} \harmonicSum['] \conj{\harmonic[']{(\rotation{\omega}g)}} \pixel[']{\conj{\harmonic[']{Y}}} \nonumber{} \\
    %
     & = \harmonicSum \harmonic{f} \conj{\harmonic{(\rotation{\omega}g)}} \nonumber                                                                                                          \\
    %
     & = \harmonicSum \harmonic{f} \conj{\big(\wigner{\ell}{m}{0}(\phi,\theta,0) \axiHarmonic{g}\big)} \nonumber{}                                                                           \\
    %
     & = \harmonicSum \sqrt{\inverseFactor} \harmonic{f} \conj{\axiHarmonic{g}} \pixel{\harmonic{Y}},
\end{align}
%
where the last line follows by \cref{eq:chapter2_wigner_axisymmetric}.
The isotropic convolution has the following properties:
%
\begin{enumerate}[(i),nosep,left=\parindent]
    \item does not support directional kernels since \(\pixel{g}\) must be axisymmetric;
    \item an output that remains on the sphere; and
    \item efficient computation since it is a product in harmonic space.
\end{enumerate}

\subsubsection{Left Convolution}

The definition of the left convolution~\cite{Kennedy2011,Driscoll1994} in real space is
%
\begin{equation}
    \pixel{(f \circleddash g)}
    %
    = \integrateRotation{\rho} f(\rotationMatrix\eta) g(\rotationMatrix^{-1}\omega),
\end{equation}
%
where \({\eta}\) is the north pole, and \(\rotationVolume=\sin{\beta} \dd{\alpha} \dd{\beta} \dd{\gamma}\) is the usual invariant measure on \(\rotationGroup{}\).
The harmonic representation of this convolution is
%
\begin{equation}
    \harmonic{(f \circleddash g)}
    %
    = 2\pi \sqrt{\inverseFactor} \harmonic{f} \axiHarmonic{g}.
\end{equation}
%
To prove this, first consider the forward harmonic transform
%
\begin{align}
    \harmonic{(f \circleddash g)}
    %
     & = \integrateSphere{\omega} \integrateRotation{\rho} f(\rotationMatrix\eta) g(\rotationMatrix^{-1}\omega) \pixel{\conj{\harmonic{Y}}} \nonumber{}                                            \\
    %
     & = \integrateRotation{\rho} f(\rotationMatrix\eta) \integrateSphere{\omega} \pixel{(\rotation{\rho}g)} \pixel{\conj{\harmonic{Y}}} \nonumber{}                                               \\
    %
     & = \integrateRotation{\rho} f(\rotationMatrix\eta) \integrateSphere{\omega} \harmonicSum['] \harmonic[']{(\rotation{\rho}g)} \pixel{\harmonic[']{Y}} \pixel{\conj{\harmonic{Y}}} \nonumber{} \\
    %
     & = \integrateRotation{\rho} f(\rotationMatrix\eta) \harmonic{(\rotation{\rho}g)} \nonumber{}                                                                                                 \\
    %
     & = \wignerSum g_{\ell n} \integrateRotation{\rho} f(\rotationMatrix\eta) \wigner{\ell}{m}{n}(\rho),
\end{align}
%
where the second and last line follow by \cref{eq:chapter2_rotation_operator,eq:chapter2_rotation_matrix} respectively.
Due to the invariant measure \(\rotationVolume{}\) and considering a right rotation about the \(z\)-axis by \({\chi}\), \ie{} \(\rho = (\alpha,\beta,\gamma) \rightarrow \rho' = (\alpha,\beta,\gamma+\chi)\), it follows
%
\begin{align}\label{eq:chapter3_so3_integral}
    \integrateRotation{\rho} f(\rotationMatrix\eta) \wigner{\ell}{m}{n}(\rho)
    %
     & = \integrateRotation{\rho} f(\rotationMatrix[']\eta) \wigner{\ell}{m}{n}(\rho') \nonumber{}            \\
    %
     & = \integrateRotation{\rho} f(\rotationMatrix\eta) \exp(-i n\chi) \wigner{\ell}{m}{n}(\rho) \nonumber{} \\
    %
     & = \exp(-i n\chi) \integrateRotation{\rho} f(\rotationMatrix\eta) \wigner{\ell}{m}{n}(\rho),
\end{align}
%
where the penultimate line follows since a rotation of \({\eta}\) about the \(z\)-axis leaves \({\eta}\) unchanged, \ie{} \(f(\rotationMatrix[']\eta) = f(\rotationMatrix\eta)\).
Thus, as \cref{eq:chapter3_so3_integral} holds for all \(\chi{}\), it must be the case that \(n=0\) unless the integral is zero, \ie{}
%
\begin{align}
    \harmonic{(f \circleddash g)}
    %
     & = \axiHarmonic{g} \integrateRotation{\rho} f(\rotationMatrix\eta) \wigner{\ell}{m}{0} (\alpha,\beta,0) \nonumber{}                                    \\
    %
     & = \axiHarmonic{g} \int\limits_{0}^{2\pi} \dd{\gamma} \integrateSphere{\omega} \pixel{f} \sqrt{\inverseFactor} \pixel{\conj{\harmonic{Y}}} \nonumber{} \\
    %
     & = 2\pi \sqrt{\inverseFactor} \harmonic{f} \axiHarmonic{g},
\end{align}
%
where the penultimate line follows by \cref{eq:chapter2_wigner_axisymmetric}.
As the harmonic representations suggest, the isotropic and left convolutions are closely related, as elaborated in~\cite{Kennedy2011}.
Hence, the properties are similar.
The left convolution has the following properties:
%
\begin{enumerate}[(i),nosep,left=\parindent]
    \item does not support directional kernels since \(\pixel{g}\) must be axisymmetric;
    \item an output that remains on the sphere; and
    \item efficient computation since it is a product in harmonic space.
\end{enumerate}

\subsubsection{Directional Convolution}

Rotations on the sphere are the spherical counterpart of translations in the Euclidean domain in real space.
Hence, the standard directional convolution is
%
\begin{equation}\label{eq:chapter3_directional_convolution}
    (f \circledast g)(\rho)
    %
    = \integrateSphere{\omega} \pixel{f} \pixel{\conj{(\rotation{\rho}g)}}.
\end{equation}
%
Upon expanding in harmonic space, this becomes (\eg{}~\cite{McEwen2007,Wandelt2001})
%
\begin{equation}
    (f \circledast g) (\rho)
    %
    = \harmonicSum \harmonic{f} \wignerSum \conj{\big(\wigner{\ell}{m}{n}(\rho) g_{\ell n}\big)},
\end{equation}
%
as
%
\begin{align}
    (f \circledast g)(\rho)
    %
     & = \integrateSphere{\omega} \pixel{f} \pixel{\conj{(\rotation{\rho}g)}} \nonumber{}                                                                                           \\
    %
     & = \integrateSphere{\omega} \harmonicSum \harmonic{f} \pixel{\harmonic{Y}} \harmonicSum['] \conj{\harmonic[']{(\rotation{\rho}g)}} \pixel{\conj{\harmonic[']{Y}}} \nonumber{} \\
    %
     & = \harmonicSum \harmonic{f} \conj{\harmonic{(\rotation{\rho}g)}} \nonumber{}                                                                                                 \\
    %
     & = \harmonicSum \harmonic{f} \wignerSum \conj{\big(\wigner{\ell}{m}{n}(\rho) g_{\ell n}\big)},
\end{align}
%
and hence, the output is on \(\rotationGroup{}\).
Fast algorithms exist~\cite{McEwen2007,Wandelt2001,Wiaux2007,McEwen2013}, but the convolution remains less efficient than a spherical harmonic transform.
The directional convolution has the following properties:
%
\begin{enumerate}[(i),nosep,left=\parindent]
    \item does support directional kernels;
    \item an output that does not remain on the sphere due to the three-dimensional rotation of the kernel; and
    \item expensive computation.
\end{enumerate}

\subsubsection{Commutative Anisotropic Convolution}

The definition of the commutative anisotropic convolution~\cite{Sadeghi2012,Khalid2012} is
%
\begin{equation}
    \pixel{(f \oplus g)}
    %
    = \integrateSphere{\omega'} \pixel[']{(\rotation{(\phi,\theta,\pi-\phi)}f)} \pixel[']{g},
\end{equation}
%
which on expansion reads
%
\begin{equation}
    \pixel{(f \oplus g)}
    %
    = \harmonicSum \wignerSum \wigner{\ell}{m}{n}(\phi,\theta,\pi-\phi) f_{\ell n} \conj{\harmonic{g}},
\end{equation}
%
as
%
\begin{align}
    \pixel{(f \oplus g)}
    %
     & = \integrateSphere{\omega'} \pixel[']{(\rotation{(\phi,\theta,\pi-\phi)}f)} \pixel[']{g} \nonumber{}                                                                                                  \\
     %
     & = \integrateSphere{\omega'} \pixel[']{(\rotation{(\phi,\theta,\pi-\phi)}f)} \pixel[']{\conj{g}} \nonumber{}                                                                                                  \\
    %
     & = \integrateSphere{\omega'} \harmonicSum \harmonic{(\rotation{(\phi,\theta,\pi-\phi)}f)} \pixel[']{\harmonic{Y}} \harmonicSum['] \conj{\harmonic[']{g}} \pixel[']{\conj{\harmonic[']{Y}}} \nonumber{} \\
    %
     & = \harmonicSum \harmonic{(\rotation{(\phi,\theta,\pi-\phi)}f)} \conj{\harmonic{g}} \nonumber{}                                                                                                        \\
    %
     & = \harmonicSum \wignerSum \wigner{\ell}{m}{n}(\phi,\theta,\pi-\phi) f_{\ell n} \conj{\harmonic{g}},
\end{align}
%
where the second line follows for real \(\pixel{g}\).
The limitation here is that one must specify the initial rotation as \({\gamma=\pi-\alpha}\) in order for the convolution to be commutative.
The complexity of the convolution is \(\mathcal{O}(L^{3}\log{L})\), and hence, it is less efficient than a spherical harmonic transform.
The convolution has the following properties:
%
\begin{enumerate}[(i),nosep,left=\parindent]
    \item it supports directional kernels;
    \item an output that remains on the sphere; and
    \item expensive computation.
\end{enumerate}

\subsection{Problem Formulation}

\cref{tab:chapter3_properties} presents a summary of the spherical convolutions discussed and their properties.
No existing definition of a spherical convolution has all the desired properties discussed in \cref{sec:chapter3_spherical_convolutions}.
In this chapter, the sifting convolution, which satisfies all desirable properties, is presented.

\begin{table}
	\centering
	\caption{
		Properties of spherical convolutions.
	}\label{tab:chapter2_properties}
	\begin{tabular}{@{}rccc@{}}
		\toprule
		                        & Anisotropic & \(\twoSphere{}\) Output & Efficient \\
		\midrule
		Isotropic               & \ding{55}   & \ding{51}               & \ding{51} \\
		Left                    & \ding{55}   & \ding{51}               & \ding{51} \\
		Directional             & \ding{51}   & \ding{55}               & \ding{55} \\
		Commutative Anisotropic & \ding{51}   & \ding{51}               & \ding{55} \\
		Sifting (this work)     & \ding{51}   & \ding{51}               & \ding{51} \\
		\bottomrule
	\end{tabular}
\end{table}


\section{Sifting Convolution}\label{sec:chapter3_sifting_convolution}

This chapter defines the sifting convolution which permits directional kernels, whose output remains on the sphere and is efficient to compute.
Moreover, it is commutative up to a complex conjugate.
The sifting convolution is constructed using a novel translation operator defined on the sphere.

\subsection{Translation Operator}\label{sec:chapter3_translation_operator}

In real space, the rotation operator on the sphere is the usual analogue of the translation operator in the Euclidean setting.
One may define an alternative operator, \(\translation{\omega}\), which follows as the analogue of the Euclidean setting but in harmonic space.
This translation is in contrast to the standard rotation as it considers two angles rather than three (\cf{} \cref{fig:chapter2_rotation_sphere}) and thereby its output remains on the sphere.
In practice, the translation operator is defined as a product of basis functions.
In the Euclidean setting, \eg{} \(\mathbb{R}\), the complex exponentials
%
\begin{equation}\label{eq:chapter3_complex_exponentials}
    \zeta_{u}(x)
    %
    = \exp(i u x),
\end{equation}
%
with \(x,\ u \in \mathbb{R}\) form the standard orthonormal basis.
A shift of coordinates defines the translation of the basis functions
%
\begin{equation}\label{eq:chapter3_exponentials_shift}
    \zeta_{u}(x + y)
    %
    = \zeta_{u}(y) \zeta_{u}(x),
\end{equation}
%
with \(y \in \mathbb{R}\) and where the final equality follows by the standard rule for exponents.
The definition of the translation of the spherical harmonics on the sphere follows by analogy with the representation as a product of basis functions
%
\begin{equation}\label{eq:chapter3_translation_basis_functions}
    \pixel{(\translation{\omega'}\harmonic{Y})}
    %
    \equiv \pixel[']{\harmonic{Y}} \pixel{\harmonic{Y}},
\end{equation}
%
where \(\omega'=(\theta',\phi')\).

This leads to a natural harmonic expression for the translation of a general arbitrary function \(f \in \hilbert{\twoSphere}\)
%
\begin{equation}\label{eq:chapter3_translation_real}
    \pixel{(\translation{\omega'}f)}
    %
    = \harmonicSum \harmonic{f} \pixel[']{\harmonic{Y}} \pixel{\harmonic{Y}},
\end{equation}
%
implying
%
\begin{equation}\label{eq:chapter3_translation_harmonic}
    \harmonic{(\translation{\omega'}f)}
    %
    = \harmonic{f} \pixel[']{\harmonic{Y}}.
\end{equation}
%
This translation operator is considered further in \cref{sec:chapter3_translation_interpretation} to build greater intuition.

\subsection{Convolution Operator}

With a translation operator to hand, one may define the sifting convolution on the sphere of \(f,\ g \in \hilbert{\twoSphere}\) in the usual manner by the inner product
%
\begin{equation}\label{eq:chapter3_convolution_real}
    \pixel{\convolution{f}{g}}
    %
    \equiv \braket*{\translation{\omega}f}{g},
\end{equation}
%
noting the use of the alternative translation operator defined in \cref{sec:chapter3_translation_operator}.
In harmonic space, this simplifies to the product
%
\begin{equation}\label{eq:chapter3_convolution_harmonic}
    \harmonic{\convolution{f}{g}}
    %
    = \harmonic{f} \conj{\harmonic{g}},
\end{equation}
%
as
%
\begin{align}
    \pixel{(f \circledcirc g)}
    %
     & = \braket*{\translation{\omega}f}{g} \nonumber{}                                                                                                                                        \\
    %
     & = \integrateSphere{\omega'} \pixel[']{(\translation{\omega}f)} \pixel[']{\conj{g}} \nonumber{}                                                                                          \\
    %
     & = \integrateSphere{\omega'} \harmonicSum \harmonic{f} \pixel[']{\harmonic{Y}} \pixel{\harmonic{Y}} \harmonicSum['] \conj{\harmonic[']{g}} \pixel[']{\conj{\harmonic[']{Y}}} \nonumber{} \\
    %
     & = \harmonicSum \harmonic{f} \conj{\harmonic{g}} \pixel{\harmonic{Y}}.
\end{align}
%
Since the harmonic representation of the convolution is simply a product (again by analogy with the harmonic representation of the Euclidean convolution), it is efficient to compute.
Note that harmonic multiplication has been considered before~\cite{Kennedy2011}; although it has been used here to define a new anisotropic convolution operator, introducing a conjugation and elaborating a real space interpretation.

\subsection{Translation Interpretation}\label{sec:chapter3_translation_interpretation}

One may show that the translation operator is simply a (sifting) convolution of a function with the shifted Dirac delta function
%
\begin{align}
    \pixel{\convolution{f}{\delta_{\omega'}}}
    %
     & = \harmonicSum \harmonic{f} \pixel[']{\harmonic{Y}} \pixel{\harmonic{Y}} \nonumber{} \\
    %
     & = \pixel{(\translation{\omega'}f)},
\end{align}
%
by noting \cref{eq:chapter3_convolution_harmonic} and where the final equality follows by \cref{eq:chapter3_translation_real}.
The sifting convolution and translation are thus natural analogues of the respective operators defined in Euclidean space.

\subsection{Properties}\label{sec:chapter3_properties}

The sifting convolution has all the desired properties discussed in \cref{sec:chapter3_spherical_convolutions}, namely, the convolution accepts directional inputs, has an output that remains on the sphere, and is efficient to compute.
\cref{tab:chapter3_properties} summarises the properties of the sifting convolution and compares them to the properties of alternative spherical convolutions.

The translation preserves symmetries, which means that any symmetry that exists in the initial kernel definition will be present after the translation.
Thus, one must be careful when choosing a kernel for a convolution to ensure that it has the desired properties when translated, \eg{} spatial localisation.
To perform anisotropic smoothing that is localised (the usual interpretation), the translated kernel also needs to be localised.
If the kernel has, say, even azimuthal symmetry, when it is translated to \(\omega'=(\theta', \phi')\) it will have a localised component both at \(\phi'\) and \(-\phi'\).
While this is not a problem \emph{per se}, for the usual interpretation of smoothing one would desire a localised component at \(\phi'\) only.
This can be achieved by ensuring that the original kernel does not exhibit a symmetry that leads to multiple localised components once translated.
The harmonic Gaussian introduced in \cref{sec:chapter3_numerical_illustration} satisfies the desired property.

\section{Numerical Illustration}\label{sec:chapter3_numerical_illustration}

This section demonstrates the effect of the sifting convolution through the application of a directional kernel to an example signal on the sphere.
Some potential kernels are defined in \cref{sec:chapter3_kernels}, where the effect of the translation is explored.
The convolution is then demonstrated with one of these kernels in \cref{sec:chapter3_convolution} on a topographic map of the Earth.
All later computations use the \texttt{SSHT} code~\cite{McEwen2011}.

\subsection{Kernels}\label{sec:chapter3_kernels}

As discussed in \cref{sec:chapter3_properties}, the translation preserves symmetries from the initial kernel definition.
Hence, one must consider the desired properties in the translated kernel, \eg{} spatial localisation.

\subsubsection{Gaussian}

First, consider an axisymmetric Gaussian in \(\ell{}\) on the sphere, \ie{}
%
\begin{equation}\label{eq:chapter3_gaussian}
    \harmonic{\mathcal{G}}
    %
    = \exp(-\frac{{\ell}^{2}}{2\sigma^{2}}),
\end{equation}
%
where \(\sigma{}\) is the standard deviation of the Gaussian.
\cref{fig:chapter3_gaussian} presents the Gaussian kernel (left panel) and the result of the translation (right panel); where the axisymmetric symmetry is preserved.
This function has no directional components and is therefore only useful for some convolution purposes.

\begin{figure}[htpb]
	\centering\capstart{}
	\subfloat[\(\Re{\pixel{\mathcal{G}}}\)]{\includegraphics[trim={23 7 3 6},clip,width=.5\textwidth]{gaussian_1000sig_L128_res512_real_norm.pdf}}
	\hfill
	\subfloat[\(\Re{\pixel{(\translation{\omega'}\mathcal{G})}}\)]{\includegraphics[trim={23 7 3 6},clip,width=.5\textwidth]{gaussian_1000sig_L128_translate_alpha3pi4_beta1pi8_res512_real_norm.pdf}}
	\caption[
		A Gaussian on the north pole and then translated
	]{
		Panel (a) presents the standard axisymmetric Gaussian on the north pole (bandlimited at \(L=128\)).
		The Gaussian is then translated to some \(\omega'=(\theta',\phi')\), as shown in panel (b).
		Plainly, the axisymmetric symmetry has been preserved under translation.
		The colour is between zero and one, reflecting the scaled intensity of the field.
	}\label{fig:chapter3_gaussian}
\end{figure}


\subsubsection{Squashed Gaussian}

Here the effect of the translation on signals with odd symmetry is considered.
One may define a \emph{squashed Gaussian} in real space as the standard Gaussian (\ie{} in the polar angle \(\theta{}\)) modulated by a sinusoidal function in the azimuthal angle \(\phi{}\) by
%
\begin{equation}\label{eq:chapter3_squashed_gaussian}
    \pixel{\mathcal{SG}}
    %
    = \exp\Bigg(-\frac{1}{2}{\bigg(\frac{\theta-\mean{\theta}}{\sigma_{\theta}}\bigg)}^{2}\Bigg)
    %
    \sin{(\nu_{\phi}\phi)},
\end{equation}
%
where \(\mean{\theta}\)/\(\sigma_{\theta}\) and \(\nu_{\phi}\) are parameters controlling the mean/standard deviation of the Gaussian and the frequency of the sine wave respectively.
The squashed Gaussian kernel is shown in the left panel of \cref{fig:chapter3_squashed_gaussian}, and the translated kernel is shown in the right panel.
Under translation, the odd symmetry in the kernel definition is preserved; and hence, the squashed Gaussian is not suitable for smoothing purposes.

\begin{figure}[htpb]
	\centering\capstart{}
	\subfloat[\(\Re\big\{\pixel{\mathcal{SG}}\big\}\)]  % chktex 21
	{\includegraphics[trim={4 7 3 6},clip,width=.5\textwidth]{squashed_gaussian_1tsig_1freq10_L128_res512_real_norm.pdf}}
	\hfill
	\subfloat[\(\Re\big\{\pixel{(\translation{\omega'}\mathcal{SG})}\big\}\)]  % chktex 21
	{\includegraphics[trim={4 7 3 6},clip,width=.5\textwidth]{squashed_gaussian_1tsig_1freq10_L128_translate_alpha3pi4_beta1pi8_res512_real_norm.pdf}}
	\caption[
		A squashed Gaussian on the north pole and then translated
	]{
		Panel (a) presents the squashed Gaussian on the north pole (bandlimited at \(L=128\)), \hl{where \mbox{\((\mean{\theta},\sigma_{\theta},\nu_{\phi}) = (0,10^{0},10^{-1})\)}, \cf{} \mbox{\cref{eq:chapter3_squashed_gaussian}}}.
		The squashed Gaussian is then translated to some \(\omega'=(\theta',\phi')\), as shown in panel (b).
		The odd azimuthal symmetry in the initial kernel definition is preserved under translation.
		The colour is between zero and one, reflecting the scaled intensity of the field.
	}\label{fig:chapter3_squashed_gaussian}
\end{figure}


\subsubsection{Elongated Gaussian}

The effect of the translation on signals with even symmetry is now considered.
One may define an \emph{elongated Gaussian} as a two-dimensional Gaussian in real space in the polar angle \(\theta{}\) and the azimuthal angle \(\phi{}\) by
%
\begin{equation}\label{eq:chapter3_elongated_gaussian}
    \pixel{\mathcal{EG}}
    %
    = \exp\bigg(-\bigg(\frac{{(\theta-\mean{\theta})}^{2}}{2\sigma_{\theta}^{2}}
        %
        + \frac{{(\phi-\mean{\phi})}^{2}}{2\sigma_{\phi}^{2}}\bigg)\bigg),
\end{equation}
%
where \((\mean{\theta},\mean{\phi})\) are the means, and \((\sigma_{\theta},\sigma_{\phi})\) are the standard deviations of the respective Gaussians.
The left panel of \cref{fig:chapter3_elongated_gaussian} presents the elongated Gaussian, and the right panel shows the result of the translation.
The elongated Gaussian has even azimuthal symmetry which results in two localised components at \(\phi'\) and \(-\phi'\) upon translation.
Thus, this kernel is not suitable for anisotropic smoothing.

\begin{figure}[htpb]
	\centering\capstart{}
	\subfloat[\(\Re{\pixel{\mathcal{EG}}}\)]{\includegraphics[trim={23 7 3 6},clip,width=.5\textwidth]{elongated_gaussian_1tsig_1psig1000_L128_res512_real_norm.pdf}}
	\hfill
	\subfloat[\(\Re{\pixel{(\translation{\omega'}\mathcal{EG})}}\)]{\includegraphics[trim={23 7 3 6},clip,width=.5\textwidth]{elongated_gaussian_1tsig_1psig1000_L128_translate_alpha3pi4_beta1pi8_res512_real_norm.pdf}}
	\caption[
		An elongated Gaussian on the north pole and then translated
	]{
		Panel (a) presents the elongated Gaussian on the north pole (bandlimited at \(L=128\)) --- where the central bar in the plot extends over half of the sphere.
		The elongated Gaussian is then translated to some \(\omega'=(\theta',\phi')\), as shown in panel (b).
		The even azimuthal symmetry in the initial kernel definition results into the two localised components at \(\phi'\) and \(-\phi'\) under translation.
		The colour is between zero and one, reflecting the scaled intensity of the field.
	}\label{fig:chapter3_elongated_gaussian}
\end{figure}


\subsubsection{Harmonic Gaussian}

One seeks a signal with neither odd nor even symmetry.
One may define a \emph{harmonic Gaussian} as a two-dimensional Gaussian in harmonic space by
%
\begin{equation}\label{eq:chapter3_harmonic_gaussian}
    \harmonic{f}
    %
    = \exp(-\bigg(\frac{{\ell}^{2}}{2\sigma_{\ell}^{2}} + \frac{{m}^{2}}{2\sigma_{m}^{2}}\bigg)).
\end{equation}
%
In effect, this function is an axisymmetric Gaussian in \(\ell{}\) modulated by a Gaussian in \(m\).
Note this function is not real --- if required one can define only the positive \(m\) components and impose reality by the conjugate symmetry relationship in harmonic space.
The function is directional, and hence, is useful in illustrating the effect of the sifting convolution on the sphere.
Consider two differently sized harmonic Gaussians on the sphere to see the effect on the sifting convolution.
\cref{fig:chapter3_harmonic_gaussian} shows both an elongated (top-left panel) and symmetric (top-right panel) harmonic Gaussians on the north pole, with the result of the corresponding translations in the bottom-left and bottom-right panels respectively.
In contrast to the other kernels, the harmonic Gaussian seems the most appropriate for anisotropic smoothing as it translates to a single point.

\begin{figure}[htpb]
\centering\capstart{}
\subfloat[\(\Re\big\{\pixel{f_{A}}\big\}\)] % chktex 21
{\includegraphics[trim={4 7 3 6},clip,width=.5\textwidth]{harmonic_gaussian_100lsig_10msig_L128_res512_real_norm.pdf}}
\hfill
\subfloat[\(\Re\big\{\pixel{f_{B}}\big\}\)]  % chktex 21
{\includegraphics[trim={4 7 3 6},clip,width=.5\textwidth]{harmonic_gaussian_10lsig_10msig_L128_res512_real_norm.pdf}}
\newline
\subfloat[\(\Re\big\{\pixel{(\translation{\omega'}f_{A})}\big\}\)]  % chktex 21
{\includegraphics[trim={4 7 3 6},clip,width=.5\textwidth]{harmonic_gaussian_100lsig_10msig_L128_translate_alpha3pi4_beta1pi8_res512_real_norm.pdf}}
\hfill
\subfloat[\(\Re\big\{\pixel{(\translation{\omega'}f_{B})}\big\}\)]  % chktex 21
{\includegraphics[trim={4 7 3 6},clip,width=.5\textwidth]{harmonic_gaussian_10lsig_10msig_L128_translate_alpha3pi4_beta1pi8_res512_real_norm.pdf}}
\caption[
Two harmonic Gaussians on the north pole and then translated
]{
The top row presents a harmonic Gaussian on the north pole (bandlimited at \(L=128\)) for two different \((\sigma_{\ell},\sigma_{m})\) values, \cf{} \cref{eq:chapter3_harmonic_gaussian}.
Panel (a) corresponds to a more elongated kernel \(f_{A}\), where \((\sigma_{\ell},\sigma_{m}) = (10^{2},10^{1})\).
The harmonic Gaussian is translated to some \(\omega'=(\theta',\phi')\) as shown in panel (c).
Whereas panel (b) corresponds to a more symmetric kernel \(f_{B}\), where \((\sigma_{\ell},\sigma_{m}) = (10^{1},10^{1})\), with the corresponding translated function in panel (d).
The colour is between zero and one, reflecting the scaled intensity of the field.
}\label{fig:chapter3_harmonic_gaussian}
\end{figure}


\subsection{Convolution}\label{sec:chapter3_convolution}

To study the effect of the sifting convolution, consider the Earth Gravitational Model EGM2008 dataset~\cite{Pavlis2013}.
This dataset is the topographic map of the Earth.
\cref{fig:chapter3_earth} presents the dataset up to an order of \(L=128\).
The sifting convolution is then performed between the Earth representation and the harmonic Gaussian, with the resultant plot given in \cref{fig:chapter3_convolved}.
As expected, when the elongated kernel is considered, as shown in the left panel, the result exhibits greater anisotropic smoothing than when considering the symmetric kernel, as shown in the right panel.
The sifting convolution supports directional kernels to perform anisotropic filtering (smoothing), while the output remains on the sphere.

\begin{figure}[htp]
	\centering
	\includegraphics[trim={23 7 3 6},clip,width=.5\textwidth]{earth_L128_res512_real_norm.pdf}
	\caption[
		The EGM2008 dataset
	]{
		EGM2008 dataset centred on a view of South America (bandlimited at \(L=128\)).
		The colour is between zero and one, reflecting the scaled intensity of the field.
	}\label{fig:chapter2_earth}
\end{figure}


\begin{figure}[htpb]
\centering\capstart{}
\subfloat[\(\Re\big\{\pixel{\convolution{f_{A}}{g}}\big\}\)]{\includegraphics[trim={4 7 3 6},clip,width=.5\textwidth]  % chktex 21
{harmonic_gaussian_100lsig_10msig_L128_convolved_earth_L128_res512_real_norm.pdf}}
\hfill
\subfloat[\(\Re\big\{\pixel{\convolution{f_{B}}{g}}\big\}\)]  % chktex 21
{\includegraphics[trim={4 7 3 6},clip,width=.5\textwidth]{harmonic_gaussian_10lsig_10msig_L128_convolved_earth_L128_res512_real_norm.pdf}}
\caption[
Two harmonic Gaussians convolved with a map of the Earth
]{
The real part of the sifting convolution between the EGM2008 dataset and the harmonic Gaussian, rotated to a view of South America (bandlimited at \(L=128\)).
Panel (a) corresponds to a more elongated kernel \(f_{A}\), where \((\sigma_{\ell},\sigma_{m}) = (10^{2}, 10^{1})\); whereas panel (b) corresponds to a more symmetric kernel \(f_{B}\), where \((\sigma_{\ell},\sigma_{m}) = (10^{1}, 10^{1})\).
As expected, the resultant sifting convolved Earth map exhibits greater anisotropic smoothing in panel (a) than in panel (b).
The sifting convolution supports directional kernels to perform anisotropic filtering (smoothing), while the output remains on the sphere.
The colour is between zero and one, reflecting the scaled intensity of the field.
}\label{fig:chapter3_convolved}
\end{figure}


\section{Conclusions}\label{sec:chapter3_conclusions}

This chapter presents the sifting convolution on the sphere and demonstrates its application.
The convolution accepts directional functions as inputs, has an output that remains on the sphere, and is efficient to compute.
The sifting convolution is defined in the usual manner through the inner product but with an alternative translation operator on the sphere.
This follows by analogy with the Euclidean translation when viewed as a convolution with a shifted Dirac delta function.
An illustration of the sifting convolution on the topographic map of the Earth demonstrates that it supports directional kernels to perform anisotropic filtering, while its output remains on the sphere.
Convolutions are an important part of signal processing techniques, hence, the sifting convolution can play an integral role in constructions of alternate spherical analysis techniques, which is the focus of \cref{sec:chapter4}.
