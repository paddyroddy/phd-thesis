\begin{appendices}

\chapter{Colophon}\label{sec:appendix}

This document was set in the Helvetica typeface using \LaTeX{} and Bib\LaTeX{}, composed with a text editor.
The typesetting software used was the Lua\TeX{} distribution and the bibliography backend used was \texttt{biber}.
This work was built on a template created by Russel Winder, maintained and distributed by Ian Kirker\footnote{\url{https://github.com/UCL/ucl-latex-thesis-templates}}.

\cref{fig:chapter4_region,fig:chapter5_region} have been plotted with \texttt{PGF} and \texttt{Ti\emph{k}Z}.
\cref{fig:chapter2_ricker_wavelets,fig:chapter2_tiling,fig:chapter2_slepian_colatitude,fig:chapter2_polar_cap_eigenvalues,fig:chapter4_tiling,fig:chapter4_eigenvalues,fig:chapter5_tiling,fig:chapter5_slepian_eigenvalues} have been created with \texttt{Matplotlib} and \texttt{seaborn}.
The majority of the remaining figures were produced with \texttt{plotly}, \ie{} \cref{fig:chapter2_spherical_harmonics,fig:chapter2_rotation_sphere,fig:chapter2_axisymmetric_wavelets,fig:chapter2_slepian_polar_cap,fig:chapter3_gaussian,fig:chapter3_squashed_gaussian,fig:chapter3_elongated_gaussian,fig:chapter3_harmonic_gaussian,fig:chapter3_earth,fig:chapter3_convolved,fig:chapter4_south_america_region,fig:chapter4_eigenfunctions,fig:chapter4_slepian_wavelets,fig:chapter4_slepian_wavelet_coefficients,fig:chapter4_denoising,fig:chapter5_eigenhomers,fig:chapter5_homer_region,fig:chapter5_slepian_functions,fig:chapter5_wavelets,fig:chapter5_homer_data,fig:chapter5_wavelet_coefficients,fig:chapter5_denoising,fig:chapter5_other_meshes}.
The \texttt{SSHT}\footnote{\url{http://astro-informatics.github.io/ssht/}}~\cite{McEwen2011} and \texttt{S2LET}\footnote{\url{http://astro-informatics.github.io/s2let/}}~\cite{Leistedt2013} codes have been used to compute the spherical harmonic transforms and the wavelet tiling, respectively.
To construct the meshes and compute the respective mesh Laplacians in \cref{sec:chapter5}, the \texttt{libigl}\footnote{\url{https://libigl.github.io/}}~\cite{Libigl2017} code has been leveraged.
Finally, the \texttt{S2SLEPLET}\footnote{\url{https://github.com/astro-informatics/s2sleplet}} code has been developed during the PhD to compute the Slepian wavelets.

% description of document, e.g. type faces, TeX used, TeXmaker, packages and things used for figures. Like a computational details section.
% e.g. http://tex.stackexchange.com/questions/63468/what-is-best-way-to-mention-that-a-document-has-been-typeset-with-tex#63503

% Side note:
%http://tex.stackexchange.com/questions/1319/showcase-of-beautiful-typography-done-in-tex-friends

\end{appendices}
