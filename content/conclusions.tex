\chapter{Conclusions}\label{sec:chapter6}

Many branches of science and engineering measure data on non-Euclidean manifolds.
Arguably the most well studied manifold is the sphere which occurs in fields such as astrophysics and cosmology.
Whilst many signal processing methods were initially developed in the one-dimensional Euclidean setting, it is of increasing important that these methods are generalised to geometries such as the sphere.
Due to recent interest in geometric deep learning, many methods may be generalised further to handle manifold and graph data.
Sometimes data are not observed over the whole manifold and as such whole manifold methods may not be appropriate for accurate analysis.
In this thesis, a new wavelet basis has been developed which is built on the eigenfunctions of the Slepian concentration problem.
These eigenfunctions and indeed subsequent wavelets, are optimally concentrated within a given region.
The so-called Slepian wavelets developed here may find use in many applications such as in analyses of the cosmic microwave background, in which foreground emissions dominate the central region of the spherical data.
