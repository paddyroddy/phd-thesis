\chapter{Conclusions}\label{sec:chapter6}

Many branches of science and engineering measure data on non-Euclidean manifolds.
Arguably the most well-studied manifold is the sphere, which occurs in fields such as astrophysics and cosmology.
Whilst many signal processing methods were initially developed in the one-dimensional Euclidean setting, it is of increasing importance that these methods are generalised to geometries such as the sphere.
Due to recent interest in geometric deep learning, many methods may be generalised further to handle manifold and graph data.
Sometimes data are not observed over the whole manifold and, as such, whole-manifold methods may not be appropriate for accurate analysis.
In this thesis, a new wavelet basis has been developed which is built on the eigenfunctions of the Slepian concentration problem.
These eigenfunctions and indeed ensuing wavelets are optimally concentrated within a given region.
The so-called Slepian wavelets developed here may find use in many applications, such as in analyses of the cosmic microwave background (CMB), in which foreground emissions dominate the central region of the spherical data.

An important tool in signal processing is the convolution.
Initially developed in the one-dimensional Euclidean setting, the convolution has since been extended to other geometries, such as the sphere.
However, often these convolutions do not fully emulate the Euclidean setting.
Developed in this thesis, the so-called sifting convolution has been designed to directly tackle the shortcomings of existing spherical convolutions.
In particular, the sifting convolution accepts directional functions as inputs, has an output that remains on the sphere, and is efficient to compute.
The sifting convolution is defined in the usual manner via an inner product but with a different translation operator on the sphere.
This translation operator may be viewed as a sifting convolution of a function with a shifted Dirac delta function, which follows as a natural analogue of the Euclidean definition.
A convolution is performed on a topographic map of the Earth, thus demonstrating the support of directional kernels to perform anisotropic filtering whilst the output remains on the sphere.
Although the sifting convolution was designed with the sphere in mind, the framework is entirely generic and hence may be used with any set of appropriate basis functions.

Wavelets allow one to probe spatially localised, scale-dependent features of a given signal.
Whilst wavelets have been put to great use in cosmological analyses, the addition of a mask contaminates neighbouring wavelet coefficients which overlap with the masked region.
These distorted coefficients may then be detected and removed for accurate analysis.
However, by removing the wavelet coefficients near the boundary, the power of the data is not fully utilised.
Otherwise, the missing data on the manifold may be estimated, which may then be analysed through some whole-manifold method.
Inherent uncertainty in the estimated region still exists and will persist throughout the analysis.
Instead, by leveraging the eigenfunctions of the Slepian spatial-spectral concentration problem, wavelets may be constructed within the regions of which data are present ---
thus ensuring the maximal use of the data.
Constructed in this thesis, the so-called Slepian wavelets are built on a tiling of the Slepian harmonic line using the scale-discretised framework.
A wavelet transform is developed which allows for exact reconstruction --- conditional on the accuracy of the Slepian functions on which the wavelets are built.
This wavelet transform relies on a generalisation of the sifting convolution described above to work with the Slepian functions as a basis.
The wavelets and corresponding wavelet coefficients are found of an example region, constructed from a topographic map of the Earth.
This signal is then corrupted with Gaussian white noise, and a straightforward denoising scheme produces a boost in the signal-to-nose ratio of the data.

Whilst the motivation and initial development of the methods in this thesis was developed with the sphere in mind, the framework is entirely generic.
The sifting convolution has a representation in both the real and Fourier domains and, as such, may handle any set of basis functions.
Moreover, the Slepian spatial-spectral problem may be formulated for a given region of any manifold to find the set of optimally concentrated bandlimited functions.
The Slepian wavelet construction in the manifold/graph settings is analogous to the spherical case, but where the basis functions are instead the eigenfunctions of the Laplace-Beltrami operator/eigenvectors of the graph Laplacian.
By way of demonstration, a region is constructed on a mesh of a Homer Simpson and the resulting Slepian functions are computed.
Upon construction of a suitable field on the mesh, the Slepian wavelets and corresponding wavelet coefficients are found and presented.
Following the denoising scheme developed for the spherical setting, a signal-to-noise boost is observed for the Homer mesh along with a series of other meshes.

Slepian wavelets have many potential applications in analyses of manifolds where data are only observed over a partial region.
One such application is in CMB analyses, where observations are inherently made on the celestial sphere, and foreground emissions mask the centre of the data.
In fields such as astrophysics and cosmology, datasets are increasingly large and thus require analysis at high resolutions for accurate predictions.
Whilst Slepian wavelets may be trivially computed from a set of Slepian functions, the computation of the spherical Slepian functions themselves are computationally complex, where the matrix scales as \(\mathcal{O}(L^{4})\).
Although symmetries of this matrix and the spherical harmonics have been exploited, filling in this matrix is inherently slow due to the many integrals performed.
The matrix is filled in parallel in \texttt{Python} using \texttt{multiprocessing}, and the spherical harmonic transforms are computed in \texttt{C} using \texttt{SSHT}.
This may be sped up further by utilising the new \texttt{ducc0} backend for \texttt{SSHT}, which may allow for a multithreaded solution.
Ultimately, the eigenproblem must be solved to compute the Slepian functions, requiring sophisticated algorithms to balance speed and accuracy.
Therefore, to work with high-resolution data such as these, one requires HPC methods on supercomputers with massive memory and storage.
To this end, Slepian wavelets may be exploited at present at low resolutions, but further work is required for them to be fully scalable.
