\chapter{Conclusions}\label{sec:chapter6}

Many branches of science and engineering measure data on non-Euclidean manifolds.
Arguably the most well-studied manifold is the sphere, which occurs in fields such as astrophysics and cosmology.
Whilst many signal processing methods were initially developed in the one-dimensional Euclidean setting, it is of increasing importance that these methods are generalised to geometries such as the sphere.
Due to recent interest in geometric deep learning, many methods may be generalised further to handle manifold and graph data.
Sometimes data are not observed over the whole manifold and, as such, whole manifold methods may not be appropriate for accurate analysis.
In this thesis, a new wavelet basis has been developed which is built on the eigenfunctions of the Slepian concentration problem.
These eigenfunctions and indeed subsequent wavelets are optimally concentrated within a given region.
The so-called Slepian wavelets developed here may find use in many applications, such as in analyses of the cosmic microwave background, in which foreground emissions dominate the central region of the spherical data.

An important tool in signal processing is the convolution.
Initially developed in the one-dimensional Euclidean setting, the convolution has since been successfully extended to other geometries, such as the sphere.
However, often these convolutions do not fully emulate the Euclidean setting.
Developed in this thesis, the so-called sifting convolution has been designed to directly tackle the shortcomings of existing spherical convolutions.
In particular, the sifting convolution accepts directional functions as inputs, has an output that remains on the sphere, and is efficient to compute.
The sifting convolution is defined in the usual manner via an inner product but with a different translation operator on the sphere.
This translation operator may be viewed as a sifting convolution of a function with a shifted Dirac delta function, which follows as a natural analogue of the Euclidean definition.
A convolution is performed on a topographic map of the Earth, thus demonstrating the support of directional kernels to perform anisotropic filtering whilst the output remains on the sphere.
Although the sifting convolution was designed with the sphere in mind, the framework is entirely generic and hence must be used with any appropriate basis functions.
