\chapter{Slepian Scale-Discretised Wavelets on Manifolds}\label{sec:chapter4}

\section{Introduction}

\section{Mathematical Background and Problem Formulation}

\subsection{Mathematical Preliminaries}

\subsubsection{Riemannian Manifolds}

Let \(\manifold{}\) denote a compact, smooth, connected \(d\)-dimensional Riemannian manifold without boundary contained in \(\mathbb{R}^{n}\).
The Hilbert space \(\hilbert{\manifold}\) is formed by the set of functions \(f : \manifold \to \mathbb{R}\) that are square-integrable with respect to the Riemannian volume \(\meshVolume{}\).
The geodesic distance between two points is denoted \(r(x,x')\), and the Laplace-Beltrami operator on \(\manifold{}\) is denoted \(\laplacian{}\).
Let the set of all isometries between two manifolds \(\manifold{}\) and \(\manifold'\) be denoted \(\isom{\manifold,\manifold'}\), and set \(\isom{\manifold} = \isom{\manifold,\manifold}\) to be the isometry group of \(\manifold{}\).
Similarly, \(\diff{\manifold} = \diff{\manifold,\manifold}\) is set to be the diffeomorphism group on \(\manifold{}\).

In practice, one must discretise the manifolds and represent them as graphs.
A review of graphs is presented in \cref{sec:chapter4_graphs}.

\subsubsection{Graphs}\label{sec:chapter4_graphs}

\subsubsection{Polygonal Meshes}

\begin{figure}[htp]
	\centering
	\subfloat[\(\mesh{\phi_{2}}\)]
	{\includegraphics[trim={155 12 24 6},clip,width=.25\textwidth]{chapter4/homer_rank2_lam1-325328e-03_norm.pdf}} % chktex 8
	\hfill
	\subfloat[\(\mesh{\phi_{3}}\)]
	{\includegraphics[trim={155 12 24 6},clip,width=.25\textwidth]{chapter4/homer_rank3_lam2-425344e-03_norm.pdf}} % chktex 8
	\hfill
	\subfloat[\(\mesh{\phi_{4}}\)]
	{\includegraphics[trim={155 12 24 6},clip,width=.25\textwidth]{chapter4/homer_rank4_lam2-706937e-03_norm.pdf}} % chktex 8
	\hfill
	\subfloat[\(\mesh{\phi_{5}}\)]
	{\includegraphics[trim={155 12 24 6},clip,width=.25\textwidth]{chapter4/homer_rank5_lam2-851096e-03_norm.pdf}} % chktex 8
	\newline
	\subfloat[\(\mesh{\phi_{6}}\)]
	{\includegraphics[trim={155 12 24 6},clip,width=.25\textwidth]{chapter4/homer_rank6_lam7-806422e-03_norm.pdf}} % chktex 8
	\hfill
	\subfloat[\(\mesh{\phi_{7}}\)]
	{\includegraphics[trim={155 12 24 6},clip,width=.25\textwidth]{chapter4/homer_rank7_lam1-164387e-02_norm.pdf}} % chktex 8
	\subfloat[\(\mesh{\phi_{8}}\)]
	{\includegraphics[trim={155 12 24 6},clip,width=.25\textwidth]{chapter4/homer_rank8_lam1-738974e-02_norm.pdf}} % chktex 8
	\hfill
	\subfloat[\(\mesh{\phi_{9}}\)]
	{\includegraphics[trim={155 12 24 6},clip,width=.25\textwidth]{chapter4/homer_rank9_lam1-827015e-02_norm.pdf}} % chktex 8
	\caption{
	}\label{fig:chapter4_eigenhomers}
\end{figure}


\subsection{Slepian Scale-Discretised Wavelets on the Sphere}\label{sec:chapter4_slepian_scale_discretised_wavelets_sphere}

\subsection{Problem Formulation}

This work generalises Slepian scale-discretised wavelets on the sphere, defined in \cref{sec:chapter4_slepian_scale_discretised_wavelets_sphere}, to arbitrary manifolds.

\section{Working with Manifolds}

In this section a review of the Slepian spatial-spectral concentration problem is presented in \cref{sec:chapter4_slepian_concentration_problem_manifolds}.
A review of the sifting convolution follows in \cref{sec:chapter4_sifting_convolution_manifolds} which allows one to perform convolutions on manifolds.

\subsection{Slepian Concentration Problem on Manifolds}\label{sec:chapter4_slepian_concentration_problem_manifolds}

The problem of spatial concentration of bandlimited signals (likewise spectral concentration of spacelimited signals) was first studied by Slepian, Landau and Pollak in the 1960s~\cite{Slepian1961,Landau1961,Landau1962}.
Initially for time domain signals, it has since been extended to other domains such as signals on the sphere~\cite{Roddy2021a,Simons2006,Xu1983,Wieczorek2005}.
This work considers optimally concentrated functions within a region \(R\) of a manifold \(\manifold{}\), an example of which is shown in \cref{fig:chapter4_region}.

\begin{figure}[htp]
	\centering
	\includegraphics{create_example_region_manifold.pdf}
	\caption{
		An example region \(R\) on a manifold \(\mathcal{M}\).
	}\label{fig:chapter4_region}
\end{figure}


\subsubsection{Spatial Concentration of a Bandlimited Function}

To maximise the spatial energy concentration of a bandlimited signal \(f \in \hilbert{\manifold}\), the following ratio must be maximised:
%
\begin{equation}
	\mu
	%
	= \frac{\integrateRegion{\meshVolume} \abs{\mesh{f}}^{2}}{\integrateManifold{x} \abs{\mesh{f}}^{2}},
\end{equation}
%
where spatial concentration is measured by \(0 < \mu < 1\).
In harmonic space this can be simplified to a sum
%
\begin{equation}\label{eq:chapter4_spatial_concentration_ratio}
	\mu
	%
	= \frac{\sum\limits_{i} f_{i} \sum\limits_{j} D_{i,j} f_{j}}{\meshSum \abs{f_{i}}^{2}},
\end{equation}
%
where
%
\begin{equation}
	D_{i,j}
	%
	= \integrateRegion{\meshVolume \mesh{\phi_{i}} \mesh{\phi_{j}}}
\end{equation}
%
is a square matrix including all basis functions of the manifold.
Rewriting \cref{eq:chapter4_spatial_concentration_ratio} as a matrix variational problem
%
\begin{equation}
	\mu
	%
	= \frac{\transpose{\vb{f}} \vb{D} \vb{f}}{\transpose{\vb{f}} \vb{f}},
\end{equation}
%
where \(\vb{f}\) are the harmonic coefficients of \(\mesh{f}\), and are the solutions to the eigenproblem
%
\begin{equation}
	\vb{D}\vb{f} = \mu \vb{f}.
\end{equation}

\subsubsection{Slepian Decomposition}

\begin{figure}[htp]
	\centering
	\includegraphics[trim={156 8 21 6},clip,width=.6\textwidth]{homer_region_norm.pdf}
	\caption{
		The head region (in black) chosen to compute the Slepian functions of the Homer mesh.
	}\label{fig:chapter4_homer_region}
\end{figure}


\begin{figure}[htp]
	\centering
	\subfloat[\(\mesh{S_{1}},\ \mu=1.00\)]
	{\includegraphics[trim={101 0 3 3},clip,width=.33\textwidth]{slepian_homer_rank0_lam1-000000e00_zoom.pdf}} % chktex 8
	\hfill
	\subfloat[\(\mesh{S_{10}},\ \mu=1.00\)]
	{\includegraphics[trim={101 0 3 3},clip,width=.33\textwidth]{slepian_homer_rank9_lam1-000000e00_zoom.pdf}} % chktex 8
	\hfill
	\subfloat[\(\mesh{S_{25}},\ \mu=1.00\)]
	{\includegraphics[trim={101 0 3 3},clip,width=.33\textwidth]{slepian_homer_rank24_lam1-000000e00_zoom.pdf}} % chktex 8
	\newline
	\subfloat[\(\mesh{S_{50}},\ \mu=1.00\)]
	{\includegraphics[trim={101 0 3 3},clip,width=.33\textwidth]{slepian_homer_rank49_lam1-000000e00_zoom.pdf}} % chktex 8
	\hfill
	\subfloat[\(\mesh{S_{100}},\ \mu=1.00\)]
	{\includegraphics[trim={101 0 3 3},clip,width=.33\textwidth]{slepian_homer_rank99_lam1-000000e00_zoom.pdf}} % chktex 8
	\hfill
	\subfloat[\(\mesh{S_{200}},\ \mu=1.00\)]
	{\includegraphics[trim={101 0 3 3},clip,width=.33\textwidth]{slepian_homer_rank199_lam1-000000e00_zoom.pdf}} % chktex 8
	\caption{
		The Slepian functions of the Homer head region \(\mesh{\slepian{S}}\) for \(p \in \set{1, 10, 25, 50, 100, 200}\) shown left-to-right, top-to-bottom.
		The corresponding eigenvalue \(\slepian{\mu}\) is a measure of the concentration within the given region \(R\) which remain \(\almost{1}\) for many \(p\) values before decreasing towards zero.
	}\label{fig:chapter4_slepian_functions}
\end{figure}


\begin{figure}[htp]
	\centering\capstart
	\includegraphics[width=\textwidth]{homer_slepian_eigenvalues_b1275.pdf}
	\caption[
		The Slepian eigenvalues of the Homer head region
	]{
		The eigenvalues of the Homer head region concentrated within the Shannon number \(N=329\).
		The majority of the eigenvalues are \(\almost{1}\) before decreasing rapidly towards zero around the Shannon number.
	}\label{fig:chapter4_slepian_eigenvalues}
\end{figure}


\subsection{Sifting Convolution on Manifolds}\label{sec:chapter4_sifting_convolution_manifolds}

\section{Slepian Wavelets on Manifolds}

\subsubsection{Generating Functions}

Consider the smooth generating functions defined by~\cite{Wiaux2008} to tile the Slepian line.
Consider the \(C^{\infty}\) Schwartz function with compact support on \(\interval{-1}{1}\):
%
\begin{equation}
	s(t) \equiv
	%
	\begin{cases}
		\exp(1/(t^{2}-1)), & t \in \interval{-1}{1}    \\
		%
		0,                 & t \notin \interval{-1}{1}
	\end{cases}
\end{equation}
%
for \(t \in \mathbb{R}\).
The positive real parameter \(\lambda \in \realPosParam{}\) may then be introduced to map \(s(t)\) to
%
\begin{equation}
	s_{\lambda}(t)
	%
	\equiv s\bigg(\frac{2\lambda}{\lambda-1}(t-\lambda^{-1}) - 1\bigg),
\end{equation}
%
which has compact support in \(\interval{1/\lambda}{1}\).
One can define the smoothly decreasing function \(k_{\lambda}\) by
%
\begin{equation}
	k_{\lambda}(t)
	%
	\equiv \frac{\int_{t}^{1} \dd{t'} s^{2}_{\lambda}(t')/t'}
	%
	{\int_{1/\lambda}^{1} \dd{t'} s^{2}_{\lambda}(t')/t'},
\end{equation}
%
which is unity for \(t < 1/\lambda{}\), zero for \(t > 1\), and smoothly decreasing from unity to zero for \(t \in \interval{1/\lambda}{1}\).
The wavelet generating function is defined as
%
\begin{equation}
	\kappa_{\lambda}(t)
	%
	\equiv \sqrt{k_{\lambda}(t/\lambda) - k_{\lambda}(t)},
\end{equation}
%
and the scaling function generating function is
%
\begin{equation}
	\eta_{\lambda}(t)
	%
	\equiv \sqrt{k_{\lambda}(t)}.
\end{equation}

An instinctive approach is to define the wavelets \(\slepian{\Psi}^{j}\) from the generating functions \(\kappa_{\lambda}\) to have support on \(\interval{\lambda^{j-1}}{\lambda^{j+1}}\), yielding
%
\begin{equation}
	\slepian{\Psi}^{j}
	%
	\equiv \kappa_{\lambda}\bigg(\frac{p}{\lambda^{j}}\bigg).
\end{equation}
%
For \(p \geq \lambda^{J_{0}}\) the admissibility condition \cref{eq:chapter3_admissibility} is satisfied for these wavelets, where \(J_{0}\) is the lowest wavelet scale used in the decomposition.
Modes that cannot be probed by wavelets can be extracted through the construction of a scaling function \(\Phi{}\) (\ie{} modes with \(p < \lambda^{J_{0}}\)):
%
\begin{equation}
	\slepian{\Phi}
	%
	\equiv \eta_{\lambda}\bigg(\frac{p}{\lambda^{J_{0}}}\bigg).
\end{equation}
%
Exact reconstruction can be achieved by setting \(J\) appropriately
%
\begin{equation}
	J = \lceil{} \log_{\lambda}(N)\rceil{}.
\end{equation}
%
Assuming that \(0 \leq J_{0} < J\) is satisfied, the lowest wavelet scale \(J_{0}\) is arbitrary.
The Slepian wavelets are constructed by the tiling of the Slepian line as shown in \cref{fig:chapter4_tiling}.

\begin{figure}[htp]
	\centering
	\includegraphics[width=\textwidth]{homer_slepian_tiling_b1275.pdf}
	\caption{
	}\label{fig:chapter4_tiling}
\end{figure}


\section{Numerical Illustration}

\begin{figure}[htp]
	\centering
	\subfloat[]
	{\includegraphics[trim={156 8 21 6},clip,width=.46\textwidth]{chapter4/homer_field.pdf}}
	\hfill
	\subfloat[]
	{\includegraphics[trim={101 0 3 3},clip,width=.54\textwidth]{chapter4/slepian_homer_field_zoom.pdf}}
	\caption{
	}\label{fig:chapter4_homer_data}
\end{figure}


\begin{figure}[htp]
	\centering
	\subfloat[\(\mesh{\Phi}\)]
	{\includegraphics[trim={102 3 6 2},clip,width=.33\textwidth]{chapter4/slepian_wavelets_homer_3B_2jmin_scaling_zoom.pdf}}
	\hfill
	\subfloat[\(\mesh{\Psi^{2j}}\)]
	{\includegraphics[trim={102 3 6 2},clip,width=.33\textwidth]{chapter4/slepian_wavelets_homer_3B_2jmin_2j_zoom.pdf}}
	\hfill
	\subfloat[\(\mesh{\Psi^{3j}}\)]
	{\includegraphics[trim={102 3 6 2},clip,width=.33\textwidth]{chapter4/slepian_wavelets_homer_3B_2jmin_3j_zoom.pdf}}
	\newline
	\subfloat[\(\mesh{\Psi^{4j}}\)]
	{\includegraphics[trim={102 3 6 2},clip,width=.33\textwidth]{chapter4/slepian_wavelets_homer_3B_2jmin_4j_zoom.pdf}}
	\hfill
	\subfloat[\(\mesh{\Psi^{5j}}\)]
	{\includegraphics[trim={102 3 6 2},clip,width=.33\textwidth]{chapter4/slepian_wavelets_homer_3B_2jmin_5j_zoom.pdf}}
	\hfill
	\subfloat[\(\mesh{\Psi^{6j}}\)]
	{\includegraphics[trim={102 3 6 2},clip,width=.33\textwidth]{chapter4/slepian_wavelets_homer_3B_2jmin_6j_zoom.pdf}}
	\caption{
	}\label{fig:chapter4_wavelets}
\end{figure}


\begin{figure}[htp]
	\centering
	\subfloat[\(\mesh{W^{\Phi}}\)]
	{\includegraphics[trim={101 0 3 3},clip,width=.33\textwidth]{chapter4/slepian_wavelet_coefficients_homer_3B_2jmin_scaling_zoom.pdf}}
	\hfill
	\subfloat[\(\mesh{W^{\Psi^{2j}}}\)]
	{\includegraphics[trim={101 0 3 3},clip,width=.33\textwidth]{chapter4/slepian_wavelet_coefficients_homer_3B_2jmin_2j_zoom.pdf}}
	\hfill
	\subfloat[\(\mesh{W^{\Psi^{3j}}}\)]
	{\includegraphics[trim={101 0 3 3},clip,width=.33\textwidth]{chapter4/slepian_wavelet_coefficients_homer_3B_2jmin_3j_zoom.pdf}}
	\newline
	\subfloat[\(\mesh{W^{\Psi^{4j}}}\)]
	{\includegraphics[trim={101 0 3 3},clip,width=.33\textwidth]{chapter4/slepian_wavelet_coefficients_homer_3B_2jmin_4j_zoom.pdf}}
	\hfill
	\subfloat[\(\mesh{W^{\Psi^{5j}}}\)]
	{\includegraphics[trim={101 0 3 3},clip,width=.33\textwidth]{chapter4/slepian_wavelet_coefficients_homer_3B_2jmin_5j_zoom.pdf}}
	\hfill
	\subfloat[\(\mesh{W^{\Psi^{6j}}}\)]
	{\includegraphics[trim={101 0 3 3},clip,width=.33\textwidth]{chapter4/slepian_wavelet_coefficients_homer_3B_2jmin_6j_zoom.pdf}}
	\caption{
	}\label{fig:chapter4_wavelet_coefficients}
\end{figure}


\begin{figure}[htp]
	\centering\capstart{}
	\subfloat[Initial Data]
	{\includegraphics[trim={101 0 3 3},clip,width=.33\textwidth]{slepian_homer_field_zoom.pdf}}
	\hfill
	\subfloat[Noisy Data \newline
		\(\snr{z} = \SI{0.32}{\dB}\)]
	{\includegraphics[trim={101 0 3 3},clip,width=.33\textwidth]{slepian_homer_field_-5noise_zoom.pdf}}
	\hfill
	\subfloat[Denoised \(N_{\sigma}=1\) \newline
		\(\snr{d} = \SI{2.29}{\dB}\)]
	{\includegraphics[trim={101 0 3 3},clip,width=.33\textwidth]{homer_-5snr_1n_denoised.pdf}}
	\caption[
		A denoising demonstration for a field on the Homer mesh
	]{
		Panel (a) shows the data in the region \(R\) constructed from the Slepian coefficients of the per vertex normals (\cf{} \cref{fig:chapter4_homer_data}) --- where the field value outside the region is set to negative infinity for illustrative purposes.
		Gaussian white noise is added to the signal in the Homer head region with a signal-to-noise ratio of \(\SI{0.32}{\dB}\), shown in panel (b).
		The scaling and wavelet coefficients of the noisy signal are calculated and are then hard-thresholded with \(N_{\sigma}=1\).
		The corresponding denoised plot is shown in panel (c), where the signal-to-noise ratio is boosted by \(\SI{1.97}{\dB}\) to \(\SI{2.29}{\dB}\).
		Whilst the signal values are defined on the vertices, they have been averaged onto the faces for the plot.
	}\label{fig:chapter4_denoising}
\end{figure}


\begin{figure}[htp]
	\centering
	\subfloat[Bird]
	{\includegraphics[trim={7 8 3 7},clip,width=.38\textwidth]{bird_region_norm.pdf}}
	\hfill
	\subfloat[Cheetah]
	{\includegraphics[trim={137 1 3 7},clip,width=.28\textwidth]{cheetah_region_norm.pdf}}
	\hfill
	\subfloat[Cube]
	{\includegraphics[trim={62 1 3 7},clip,width=.33\textwidth]{cube_region_norm.pdf}}
	\newline
	\subfloat[Dragon]
	{\includegraphics[trim={75 8 3 7},clip,width=.33\textwidth]{dragon_region_norm.pdf}}
	%
	\subfloat[Teapot]
	{\includegraphics[trim={3 8 3 7},clip,width=.38\textwidth]{teapot_region_norm.pdf}}
	\caption{
		The Slepian regions (in black) of some other meshes.
		The same denoising procedure as in \cref{fig:chapter4_denoising} was performed for these alternative meshes, the results are shown in \cref{tab:chapter4_denoising}.
	}\label{fig:chapter4_other_meshes}
\end{figure}


\begin{table}
	\centering
	\caption{
		Denoising of other meshes.
	}\label{tab:chapter4_denoising}
	\begin{tabular}{@{}rcccc@{}}
		\toprule
		        & Shannon       & Wavelets    & Initial SNR        & \(N_{\sigma}=1\) SNR \\
		\midrule
		Cheetah & \(\num{72}\)  & \(\num{4}\) & \(\SI{0.32}{\dB}\) & \(\SI{1.38}{\dB}\)   \\
		%
		Dragon  & \(\num{169}\) & \(\num{5}\) & \(\SI{0.32}{\dB}\) & \(\SI{1.04}{\dB}\)   \\
		%
		Bird    & \(\num{194}\) & \(\num{5}\) & \(\SI{0.32}{\dB}\) & \(\SI{1.47}{\dB}\)   \\
		%
		Teapot  & \(\num{256}\) & \(\num{6}\) & \(\SI{0.32}{\dB}\) & \(\SI{1.19}{\dB}\)   \\
		%
		Cube    & \(\num{272}\) & \(\num{6}\) & \(\SI{0.32}{\dB}\) & \(\SI{2.67}{\dB}\)   \\
		%
		Homer   & \(\num{329}\) & \(\num{6}\) & \(\SI{0.32}{\dB}\) & \(\SI{2.29}{\dB}\)   \\
		\bottomrule
	\end{tabular}
\end{table}

