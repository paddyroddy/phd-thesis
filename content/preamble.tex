% This may change in a future version,
%  but given that some people needed it, if you need a different degree title
%  (e.g. Master of Science, Master in Science, Master of Arts)
%  uncomment the following 3 lines and set as appropriate (this *has* to be before \maketitle)
% \makeatletter
% \renewcommand {\@degree@string} {Master of Things}
% \makeatother

\title{Slepian Wavelets for the Analysis of Incomplete Data on Manifolds}
\author{Patrick James Roddy}
\department{Centre for Doctoral Training in Data Intensive Science}

\maketitle
\makedeclaration{}

\begin{abstract} % 300 word limit
	Many fields in science and engineering measure data that inherently live on non-Euclidean geometries, such as the sphere.
	It is increasingly important that techniques developed in the Euclidean setting are extended to other geometries.
	Due to recent interest in geometric deep learning, analogues of Euclidean techniques must also handle general manifolds or graphs.
	Often, data are only observed over partial regions of manifolds, and thus standard whole manifold techniques may not yield accurate predictions.
	In this thesis, a new wavelet basis is designed for datasets like these.

	Although many definitions of spherical convolutions exist, none fully emulate the Euclidean definition.
	A novel spherical convolution is developed, designed to tackle the shortcomings of existing methods.
	The so-called sifting convolution exploits the sifting property of the Dirac delta and follows by the inner product of a function with the translated version of another.
	This translation operator is analogous to the Euclidean translation in harmonic space and exhibits some useful properties.
	In particular, the sifting convolution supports directional kernels; has an output that remains on the sphere; and is efficient to compute.
	The convolution is entirely generic and thus may be used with any set of basis functions.
	An application of the sifting convolution with a topographic map of the Earth demonstrates that it supports directional kernels to perform anisotropic filtering.

	Slepian wavelets are built upon the eigenfunctions of the Slepian concentration problem of the manifold --- a set of bandlimited functions which are maximally concentrated within a given region.
	Wavelets are constructed through a tiling of the Slepian harmonic line by leveraging the scale-discretised framework.
	A straightforward denoising formalism demonstrates a boost in signal-to-noise for both a spherical and manifold example.
	Whilst these wavelets were inspired by spherical datasets like in cosmology, the wavelet construction may be utilised for manifold or graph data.
\end{abstract}

\begin{impactstatement}

	UCL theses now have to include an impact statement. The following text is the description from the guide linked from the formatting and submission website of what that involves. (Link to the guide: {\scriptsize \url{http://www.grad.ucl.ac.uk/essinfo/docs/Impact-Statement-Guidance-Notes-for-Research-Students-and-Supervisors.pdf}})

	\begin{quote}
		The statement describes, in no more than 500 words, how the expertise, knowledge, analysis,
		discovery or insight presented in your thesis could be put to a beneficial use. Consider benefits both
		inside and outside academia and the ways in which these benefits could be brought about.

		The benefits inside academia could be to the discipline and future scholarship, research methods,
		the curriculum; they might be within your research area and within other
		research areas.

		The benefits outside academia could occur to commercial activity, social enterprise, professional
		practice, clinical use, public health, public policy design, public service delivery, laws, public
		discourse, culture, the quality of the environment or quality of life.

		The impact could occur locally, regionally, nationally or internationally, to individuals, communities or
		organisations and could be immediate or occur incrementally, in the context of a broader field of
		research, over many years, decades or longer.

		Impact could be brought about through disseminating outputs (either in scholarly journals or
		elsewhere such as specialist or mainstream media), education, public engagement, translational
		research, commercial and social enterprise activity, engaging with public policymakers and public
		service delivery practitioners, influencing ministers, collaborating with academics and non-academics.

		Further information including a searchable list of hundreds of examples of UCL impact outside of
		academia please see \url{https://www.ucl.ac.uk/impact/}. For thousands more examples, please see
		\url{http://results.ref.ac.uk/Results/SelectUoa}.
	\end{quote}
\end{impactstatement}

\begin{acknowledgements}
	Acknowledge all the things!
\end{acknowledgements}

\setcounter{tocdepth}{2}
% Setting this higher means you get contents entries for
%  more minor section headers.

\tableofcontents
\listoffigures
\listoftables
