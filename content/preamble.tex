% This may change in a future version,
%  but given that some people needed it, if you need a different degree title
%  (e.g. Master of Science, Master in Science, Master of Arts)
%  uncomment the following 3 lines and set as appropriate (this *has* to be before \maketitle)
% \makeatletter
% \renewcommand {\@degree@string} {Master of Things}
% \makeatother

\title{Slepian Wavelets for the Analysis of Incomplete Data on Manifolds}
\author{Patrick James Roddy}
\department{Centre for Doctoral Training in Data Intensive Science}

\maketitle
\makedeclaration{}

\begin{abstract} % 300 word limit
	Many fields in science and engineering measure data that inherently live on non-Euclidean geometries, such as the sphere.
	Techniques developed in the Euclidean setting must be extended to other geometries.
	Due to recent interest in geometric deep learning, analogues of Euclidean techniques must also handle general manifolds or graphs.
	Often, data are only observed over partial regions of manifolds, and thus standard whole-manifold techniques may not yield accurate predictions.
	In this thesis, a new wavelet basis is designed for datasets like these.

	Although many definitions of spherical convolutions exist, none fully emulate the Euclidean definition.
	A novel spherical convolution is developed, designed to tackle the shortcomings of existing methods.
	The so-called sifting convolution exploits the sifting property of the Dirac delta and follows by the inner product of a function with the translated version of another.
	This translation operator is analogous to the Euclidean translation in harmonic space and exhibits some useful properties.
	In particular, the sifting convolution supports directional kernels; has an output that remains on the sphere; and is efficient to compute.
	The convolution is entirely generic and thus may be used with any set of basis functions.
	An application of the sifting convolution with a topographic map of the Earth demonstrates that it supports directional kernels to perform anisotropic filtering.

	Slepian wavelets are built upon the eigenfunctions of the Slepian concentration problem of the manifold --- a set of bandlimited functions which are maximally concentrated within a given region.
	Wavelets are constructed through a tiling of the Slepian harmonic line by leveraging the existing scale-discretised framework.
	A straightforward denoising formalism demonstrates a boost in signal-to-noise for both a spherical and general manifold example.
	Whilst these wavelets were inspired by spherical datasets, like in cosmology, the wavelet construction may be utilised for manifold or graph data.
\end{abstract}

\begin{impactstatement} % 500 word limit
	This thesis is primarily split into two parts: the development of a spherical convolution, and the construction of a wavelet basis built on this convolution.
	The sifting convolution is designed as a Euclidean analogue; although initially designed for the sphere, it is indeed general.
	Slepian wavelets were again considered for the sphere, but may be generalised further to graph and manifold data.
	These wavelets are intended for the analysis of incomplete manifolds.

	Convolutions are a fundamental technique in signal processing that are used widely in scientific and engineering applications.
	Often the convolution is part of a wider technique, although they may be used directly in their own right.
	Existing convolutions in the spherical domain impose certain restrictions on one or both inputs.
	The sifting convolution therefore may find use in various applications where one seeks to accept directional inputs whilst remaining on the sphere.
	Moreover, the convolution is general and, as such, may find success in general manifold settings.
	With the recent interest in geometric deep learning, the sifting convolution may find application in the development of convolutional neural networks on manifolds.

	Wavelets are also widely used in various disciplines to analyse signals both in space and scale.
	Whilst many fields measure data on manifolds (\ie{} the sphere), often data are only observed on a partial region of the manifold.
	Wavelets are a typical approach to data of this form, but the wavelet coefficients which overlap with the boundary become contaminated and must be removed for accurate analysis.
	Another approach is to estimate the region of missing data and to use existing whole-manifold methods for analysis.
	However, both approaches introduce uncertainty into any analysis.
	Slepian wavelets enable one to work directly with only the data present, thus avoiding the problems discussed above.
	Possible applications of Slepian wavelets to areas of research measuring data on the partial sphere include: gravitational/magnetic fields in geodesy; ground-based measurements in astronomy; measurements of whole-planet properties in planetary science; geomagnetism of the Earth; and in analyses of the cosmic microwave background.
\end{impactstatement}

\begin{acknowledgements}
	First, I would like to express my gratitude to my supervisor Jason D.~McEwen for providing me with advice throughout the last four years.
	Although I was initially overwhelmed with the mathematical background required to undertake this work, Jason's support over whiteboard discussions proved invaluable.
	Many of the methods developed in the thesis originated from Jason's scrawled notes from some time ago, it has been a great pleasure to build on these in practice.
	I hope that, in the future, Slepian wavelets will be successfully applied to cosmology, which is Jason's principal line of research.

	I would like to thank the Centre for Doctoral Training in Data Intensive Science (CDT DIS) for providing experiences beyond a standard PhD programme.
	The first-year training was varied and gave me insights into areas of mathematical, statistical and computational methods which I had not studied before, and deepened my existing knowledge of other topics.
	In our first year, James Legg, Kelvin Mulder and I undertook a group project titled Towards Developing Intrusion Detection Techniques.
	This work was performed with aid from Matt Lewis of NCC Group and provided me with insight into how the skills developed on the courses can be applied in industry.

	Through the CDT, I undertook a six-month placement in my third year with Vortexa, which was a fast-growing SME at the time. % typos: ignore
	At Vortexa, I was a data engineer in the Signal Processing and Enrichment team (SPET) whose responsibilities largely included ingesting the data from multiple sources, and bringing it all together in a coherent and useful manner for the downstream teams.
	I would like to thank the members of the SPET team in particular: Nat Day, Arthur Degonde, Romain Guion, Tino von Stegmann, and Ed Wright, for making me feel welcome and a valued member of the team.
	Although Romain had high expectations, he was a kind and thoughtful manager who always asked for my opinion on how the team could improve their workflow.
	I met many friendly faces at Vortexa, and the placement gave me useful insight into data science and engineering in industry.

	I have been fortunate to receive support during my PhD.
	In particular, I would like to thank Zubair Khalid for providing his \texttt{MATLAB} implementation to compute the Slepian functions of a polar cap region, as well as the formulation for a limited colatitude-longitude region.
	Furthermore, I am also grateful to Fionn Fitzmaurice for proving advice on the creation of \cref{fig:chapter5_region}.

	At UCL, I have made many friends who have helped in one way or another, too many to mention explicitly.
	Through the CDT office in my first year and the astrophysics office in later years, I have had many great colleagues.
	In our first year, I recall many discussions with Tarek Allam Jr.\ on all things on the command line and good software practices --- which I now use all the time.
	I would also like to thank, in particular, Damien De Mijolla, Davide Piras, Gordon (Kai Hou) Yip, among others who were always willing to chat about my research --- whether they fully understood what I was talking about or not.
	Lastly, to RAX (the university running club) through which I met students from all over UCL, including my various wonderful housemates.

	Outside of UCL, I would also like to thank my friends from Kenilworth, Cambridge and London who have supported me in one way another.
	My thanks also to my family, in particular to my poor father Peter who has painstakingly proofread my thesis, and my uncle Dermot who has been there for academic advice.
	Thank you also to my housemates over the years, who have always been around to talk to.
\end{acknowledgements}

\setcounter{tocdepth}{2}
% Setting this higher means you get contents entries for
%  more minor section headers.

\tableofcontents
\listoffigures
\listoftables
