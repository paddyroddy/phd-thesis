% This may change in a future version,
%  but given that some people needed it, if you need a different degree title
%  (e.g. Master of Science, Master in Science, Master of Arts)
%  uncomment the following 3 lines and set as appropriate (this *has* to be before \maketitle)
% \makeatletter
% \renewcommand {\@degree@string} {Master of Things}
% \makeatother

\title{Slepian Wavelets for the Analysis of Incomplete Data on Manifolds}
\author{Patrick James Roddy}
\department{Centre for Doctoral Training in Data Intensive Science}

\maketitle
\makedeclaration{}

\begin{abstract} % 300 word limit
	Many fields in science and engineering measure data that inherently live on non-Euclidean geometries, such as the sphere.
	It is important that techniques developed in the Euclidean setting are extended to other geometries.
	Due to recent interest in geometric deep learning, analogues of Euclidean techniques must also handle general manifolds or graphs.
	Often, data are only observed over partial regions of manifolds, and thus standard whole manifold techniques may not yield accurate predictions.
	In this thesis, a new wavelet basis is designed for datasets like these.

	Although many definitions of spherical convolutions exist, none fully emulate the Euclidean definition.
	A novel spherical convolution is developed, designed to tackle the shortcomings of existing methods.
	The so-called sifting convolution exploits the sifting property of the Dirac delta and follows by the inner product of a function with the translated version of another.
	This translation operator is analogous to the Euclidean translation in harmonic space and exhibits some useful properties.
	In particular, the sifting convolution supports directional kernels; has an output that remains on the sphere; and is efficient to compute.
	The convolution is entirely generic and thus may be used with any set of basis functions.
	An application of the sifting convolution with a topographic map of the Earth demonstrates that it supports directional kernels to perform anisotropic filtering.

	Slepian wavelets are built upon the eigenfunctions of the Slepian concentration problem of the manifold --- a set of bandlimited functions which are maximally concentrated within a given region.
	Wavelets are constructed through a tiling of the Slepian harmonic line by leveraging the existing scale-discretised framework.
	A straightforward denoising formalism demonstrates a boost in signal-to-noise for both a spherical and manifold example.
	Whilst these wavelets were inspired by spherical datasets like in cosmology, the wavelet construction may be utilised for manifold or graph data.
\end{abstract}

\begin{impactstatement}
	This thesis is primarily split into two parts: the development of a spherical convolution, and the construction of wavelet basis built on this convolution.
	The sifting convolution is a convolution designed as an Euclidean analogue; although initially designed for the sphere, it is indeed general.
	Slepian wavelets were again considered for the sphere, but may be generalised further to graph and manifold data.
	These wavelets are intended for the analysis of incomplete manifolds.

	The convolution is a fundamental technique in signal processing and used widely in scientific and engineering applications.
	Often the convolution is part of a wider technique, although they may be used directly in their own right.
	Existing convolutions in the spherical domain impose certain restrictions on one or both inputs.
	The sifting convolution therefore may find use in various applications where one seeks to accept directional inputs whilst remaining on the sphere.
	Moreover, the convolution is general and as such may find success in general manifold settings.

	Wavelets are also widely used in various disciplines to analyse signals both in space and scale.
	Whilst many fields measure data on manifolds (\ie{} the sphere), often data are only observed on a partial region of the manifold.
	Wavelets are a typical approach to data of this form, but the wavelet coefficients which overlap with the boundary become contaminated and must be removed for accurate analysis.
	Another approach is to estimate the region of missing data and use existing whole manifold methods for analysis.
	However, both approaches introduce uncertainty into any analysis.
	Slepian wavelets enable one to work directly with only the data present thus avoiding problems discussed.
\end{impactstatement}

\begin{acknowledgements}
	Acknowledge all the things!
\end{acknowledgements}

\setcounter{tocdepth}{2}
% Setting this higher means you get contents entries for
%  more minor section headers.

\tableofcontents
\listoffigures
\listoftables
