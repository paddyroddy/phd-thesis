\chapter{Introduction}\label{sec:chapter1}

\section{Outline}

The remainder of this thesis is organised as follows.

\begin{hangparas}{24pt}{1}
	\textbf{\cref{sec:chapter2}} reviews the mathematical foundations on which the work performed in the remainder of this thesis relies.

	\textbf{\cref{sec:chapter3}} presents a review of some spherical convolutions in the literature, before developing a new convolution which addresses the limitations of existing convolutions.
	The so-called \emph{sifting convolution} supports directional kernels, has an output which remains on the sphere, and is efficient to compute.

	\textbf{\cref{sec:chapter4}} defines a novel spherical wavelet basis designed for incomplete spherical datasets.
	The so-called \emph{Slepian wavelets} are built on the eigenfunctions of the Slepian spatial-spectral concentration problem on the sphere.
	A scale-discretised wavelet transform is introduced, which depends on the convolution introduced in \cref{sec:chapter3}.

	\textbf{\cref{sec:chapter5}} generalises the Slepian wavelets in \cref{sec:chapter4} to Riemannian manifolds.
	Translations and, thus, convolutions are typically not well-defined on manifolds.
	By leveraging the sifting convolution (\cf{} \cref{sec:chapter3}), convolutions may be performed in Fourier space with the eigenfunctions of the Laplace-Beltrami operator.

	\textbf{\cref{sec:chapter6}} sets out some concluding remarks, summarising the work performed in thesis and outlining potential areas of future research.
\end{hangparas}
