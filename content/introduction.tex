\chapter{Introduction}\label{sec:chapter1}

Many fields in science and engineering measure data which are intrinsically non-Euclidean in nature.
The sphere, in particular, features commonly in disciplines such as geophysics, planetary science, computer graphics, signal processing, and cosmology.
The latter of these serves as the primary motivation for the work performed in this thesis.
Of increasing importance is the necessity to translate methods developed in the Euclidean domain to other geometries, such as the sphere.
Inspired by recent interest in geometric deep learning, many methods may be generalised further to other manifolds besides the sphere.

Convolutions are a fundamental technique in signal processing.
The convolution is a mathematical operation between two signals which produces a third signal which expresses how the shape of one signal is affected by the other.
Upon selection of a suitable kernel, convolutions have enjoyed success in many applications, such as: smoothing, sharpening, high pass filtering, differentiation, and edge detection.
Initially developed in the one-dimensional Euclidean domain, convolutions have since been used extensively in the two-dimensional setting, primarily for images.
In the spherical domain, however, existing definitions often do not capture all aspects of the Euclidean definition.
Moreover, such a convolution would ideally be generalisable to other manifolds.

Another important technique in signal processing is projecting a signal onto an appropriate basis.
In Fourier analyses, a one-dimensional signal may be expressed as an infinite sum of sines and cosines.
A Fourier transform allows one to go from the representation of a signal in time into a representation in frequency.
The signal may then be analysed in more detail by performing a cut in frequency space and analysing the resulting components independently.
In the spherical setting, a signal on the surface of the sphere may be expressed as a weighted sum of the spherical harmonics.
Spherical harmonic transforms follow as an analogy of Fourier transforms, whereby one can transform between real space and harmonic space.
Wavelets may be used as an alternative basis, which have been extended to other geometries in recent times.
Wavelets enable one to analyse contributions of scale-dependent features in both space and scale.
Applications of wavelets include: computer vision, compression, denoising, and time series analysis.
A wavelet transform, again analogous to the Fourier transform, inherently relies on a suitable convolution for synthesis and analysis.

Often data are only observed over a partial region of a given manifold, \ie{} the sphere.
For example, strong foreground microwave emissions are typically removed in analyses of the cosmic microwave background which are caused by the Galactic bulge of the Milky Way.
Thus, whole sphere methods may not be suitable for accurate analysis.
By leveraging the Slepian concentration problem, one may find the set of bandlimited functions which are optimally concentrated within a particular region.
A signal restricted to a given region (due to a limited set of observations) may then be decomposed into this basis as before.
Initially developed in the one-dimensional Euclidean setting as a time-frequency concentration problem, the so-called Slepian functions have since been used in spatial-spectral applications in geophysics, planetary sciences, and medical imaging.

The focus of this thesis is to build a framework that leverages both wavelets and the basis functions of the Slepian concentration problem.
The resulting set of functions, known as Slepian wavelets, thus profit from both approaches.
Whilst wavelets and Slepian functions both involve localised analysis, (spherical) wavelets involve spatial-scale analysis whilst Slepian functions perform spatial-spectral analysis.
A tiling of the Slepian harmonic line is utilised to develop Slepian scale-discretised wavelets.
The scale-discretised wavelet transform allows for an exact wavelet transform, where the only limiting factor is the precision of the Slepian functions themselves.
A spherical convolution is first developed which preserves a variety of desirable properties from the standard Euclidean definition.
However, the convolution described may be generalised to other manifolds beyond the sphere, and hence Slepian scale-discretised wavelets may also be arbitrarily extended.
Although analysis of the foreground contaminated cosmic microwave background has served as the primary motivation for this work, the resulting Slepian wavelets may be used for applications on the sphere and beyond.

\section{Outline}

The remainder of this thesis is organised as follows.

\begin{hangparas}{24pt}{1}
	\textbf{\cref{sec:chapter2}} reviews the mathematical foundations on which the work performed in the remainder of this thesis relies.

	\textbf{\cref{sec:chapter3}} presents a review of some spherical convolutions in the literature, before developing a new convolution which addresses the limitations of existing convolutions.
	The so-called sifting convolution supports directional kernels, has an output which remains on the sphere, and is efficient to compute.

	\textbf{\cref{sec:chapter4}} defines a novel spherical wavelet basis designed for incomplete spherical datasets.
	The so-called Slepian wavelets are built on the eigenfunctions of the Slepian spatial-spectral concentration problem on the sphere.
	A scale-discretised wavelet transform is introduced, which depends on the convolution introduced in \cref{sec:chapter3}.

	\textbf{\cref{sec:chapter5}} generalises the Slepian wavelets in \cref{sec:chapter4} to Riemannian manifolds.
	Translations and, thus, convolutions are typically not well-defined on manifolds.
	By leveraging the sifting convolution (\cf{} \cref{sec:chapter3}), convolutions may be performed in Fourier space with the eigenfunctions of the Laplace-Beltrami operator.

	\textbf{\cref{sec:chapter6}} sets out some concluding remarks, summarising the work performed in thesis and outlining potential areas of future research.
\end{hangparas}
